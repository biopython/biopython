\chapter{Appendix: Useful stuff about Python}
\label{sec:appendix}

If you haven't spent a lot of time programming in Python, many
questions and problems that come up in using Biopython are often
related to Python itself. This section tries to present some ideas and
code that come up often (at least for us!) while using the Biopython
libraries. If you have any suggestions for useful pointers that could
go here, please contribute!

\section{What the heck is a handle?}
\label{sec:appendix-handles}

Handles are mentioned quite frequently throughout this documentation,
and are also fairly confusing (at least to me!). Basically, you can
think of a handle as being a ``wrapper'' around text information.

Handles provide (at least) two benefits over plain text information:

\begin{enumerate}
  \item They provide a standard way to deal with information stored in
  different ways. The text information can be in a file, or in a
  string stored in memory, or the output from a command line program,
  or at some remote website, but the handle provides a common way of
  dealing with information in all of these formats.

  \item They allow text information to be read incrementally, instead
  of all at once. This is really important when you are dealing with
  huge text files which would use up all of your memory if you had to
  load them all.
\end{enumerate}

Handles can deal with text information that is being read (e.~g.~reading
from a file) or written (e.~g.~writing information to a file). In the
case of a ``read'' handle, commonly used functions are \verb|read()|,
which reads the entire text information from the handle, and
\verb|readline()|, which reads information one line at a time. For
``write'' handles, the function \verb|write()| is regularly used.

The most common usage for handles is reading information from a file,
which is done using the built-in Python function \verb|open|. Here, we open a
handle to the file \href{examples/m\_cold.fasta}{m\_cold.fasta}
(also available online
\href{http://biopython.org/DIST/docs/tutorial/examples/m\_cold.fasta}{here}):

\begin{verbatim}
>>> handle = open("m_cold.fasta", "r")
>>> handle.readline()
">gi|8332116|gb|BE037100.1|BE037100 MP14H09 MP Mesembryanthemum ...\n"
\end{verbatim}

Handles are regularly used in Biopython for passing information to parsers.
For example, since Biopython 1.54 the main functions in \verb|Bio.SeqIO|
and \verb|Bio.AlignIO| have allowed you to use a filename instead of a
handle:

\begin{verbatim}
from Bio import SeqIO
for record in SeqIO.parse("m_cold.fasta", "fasta"):
    print(record.id, len(record))
\end{verbatim}

On older versions of Biopython you had to use a handle, e.g.

\begin{verbatim}
from Bio import SeqIO
handle = open("m_cold.fasta", "r")
for record in SeqIO.parse(handle, "fasta"):
    print(record.id, len(record))
handle.close()
\end{verbatim}

This pattern is still useful - for example suppose you have a gzip
compressed FASTA file you want to parse:

\begin{verbatim}
import gzip
from Bio import SeqIO
handle = gzip.open("m_cold.fasta.gz")
for record in SeqIO.parse(handle, "fasta"):
    print(record.id, len(record))
handle.close()
\end{verbatim}

See Section~\ref{sec:SeqIO_compressed} for more examples like this,
including reading bzip2 compressed files.

\subsection{Creating a handle from a string}

One useful thing is to be able to turn information contained in a
string into a handle. The following example shows how to do this using
\verb|cStringIO| from the Python standard library:

%doctest
\begin{verbatim}
>>> my_info = 'A string\n with multiple lines.'
>>> print(my_info)
A string
 with multiple lines.
>>> from StringIO import StringIO
>>> my_info_handle = StringIO(my_info)
>>> first_line = my_info_handle.readline()
>>> print(first_line)
A string
<BLANKLINE>
>>> second_line = my_info_handle.readline()
>>> print(second_line)
 with multiple lines.
\end{verbatim}

\begin{thebibliography}{99}
\bibitem{cock2009}
Peter J. A. Cock, Tiago Antao, Jeffrey T. Chang, Brad A. Chapman, Cymon J. Cox, Andrew Dalke, Iddo Friedberg, Thomas Hamelryck, Frank Kauff, Bartek Wilczynski, Michiel J. L. de Hoon: ``Biopython: freely available Python tools for computational molecular biology and bioinformatics''. {\it Bioinformatics} {\bf 25} (11), 1422--1423 (2009). \href{http://dx.doi.org/10.1093/bioinformatics/btp163}{doi:10.1093/bioinformatics/btp163},
\bibitem{pritchard2006}
Leighton Pritchard, Jennifer A. White, Paul R.J. Birch, Ian K. Toth: ``GenomeDiagram: a python package for the visualization of large-scale genomic data''.  {\it Bioinformatics} {\bf 22} (5): 616--617 (2006).
\href{http://dx.doi.org/10.1093/bioinformatics/btk021}{doi:10.1093/bioinformatics/btk021},
\bibitem{toth2006}
Ian K. Toth, Leighton Pritchard, Paul R. J. Birch: ``Comparative genomics reveals what makes an enterobacterial plant pathogen''. {\it Annual Review of Phytopathology} {\bf 44}: 305--336 (2006).
\href{http://dx.doi.org/10.1146/annurev.phyto.44.070505.143444}{doi:10.1146/annurev.phyto.44.070505.143444},
\bibitem{vanderauwera2009}
G\'eraldine A. van der Auwera, Jaroslaw E. Kr\'ol, Haruo Suzuki, Brian Foster, Rob van Houdt, Celeste J. Brown, Max Mergeay, Eva M. Top: ``Plasmids captured in C. metallidurans CH34: defining the PromA family of broad-host-range plasmids''.
\textit{Antonie van Leeuwenhoek} {\bf 96} (2): 193--204 (2009).
\href{http://dx.doi.org/10.1007/s10482-009-9316-9}{doi:10.1007/s10482-009-9316-9}
\bibitem{proux2002}
Caroline Proux, Douwe van Sinderen, Juan Suarez, Pilar Garcia, Victor Ladero, Gerald F. Fitzgerald, Frank Desiere, Harald Br\"ussow:
``The dilemma of phage taxonomy illustrated by comparative genomics of Sfi21-Like Siphoviridae in lactic acid bacteria''.  \textit{Journal of Bacteriology} {\bf 184} (21): 6026--6036 (2002).
\href{http://dx.doi.org/10.1128/JB.184.21.6026-6036.2002}{http://dx.doi.org/10.1128/JB.184.21.6026-6036.2002}
\bibitem{jupe2012}
Florian Jupe, Leighton Pritchard, Graham J. Etherington, Katrin MacKenzie, Peter JA Cock, Frank Wright, Sanjeev Kumar Sharma1, Dan Bolser, Glenn J Bryan, Jonathan DG Jones, Ingo Hein: ``Identification and localisation of the NB-LRR gene family within the potato genome''. \textit{BMC Genomics} {\bf 13}: 75 (2012).
\href{http://dx.doi.org/10.1186/1471-2164-13-75}{http://dx.doi.org/10.1186/1471-2164-13-75}
\bibitem{cock2010}
Peter J. A. Cock, Christopher J. Fields, Naohisa Goto, Michael L. Heuer, Peter M. Rice: ``The Sanger FASTQ file format for sequences with quality scores, and the Solexa/Illumina FASTQ variants''.  \textit{Nucleic Acids Research} {\bf 38} (6): 1767--1771 (2010). \href{http://dx.doi.org/10.1093/nar/gkp1137}{doi:10.1093/nar/gkp1137}
\bibitem{brown1999}
Patrick O. Brown, David Botstein: ``Exploring the new world of the genome with DNA microarrays''. \textit{Nature Genetics} {\bf 21} (Supplement 1), 33--37 (1999). \href{http://dx.doi.org/10.1038/4462}{doi:10.1038/4462}
\bibitem{talevich2012}
Eric Talevich, Brandon M. Invergo, Peter J.A. Cock, Brad A. Chapman: ``Bio.Phylo: A unified toolkit for processing, analyzing and visualizing phylogenetic trees in Biopython''.  \textit{BMC Bioinformatics} {\bf 13}: 209 (2012).  \href{http://dx.doi.org/10.1186/1471-2105-13-209}{doi:10.1186/1471-2105-13-209}
\bibitem{cornish1985}
Athel Cornish-Bowden: ``Nomenclature for incompletely specified bases in nucleic acid sequences: Recommendations 1984.'' \textit{Nucleic Acids Research} {\bf 13} (9): 3021--3030 (1985). \href{http://dx.doi.org/10.1093/nar/13.9.3021}{doi:10.1093/nar/13.9.3021}
\bibitem{cavener1987}
Douglas R. Cavener: ``Comparison of the consensus sequence flanking translational start sites in Drosophila and vertebrates.'' \textit{Nucleic Acids Research} {\bf 15} (4): 1353--1361 (1987). \href{http://dx.doi.org/10.1093/nar/15.4.1353}{doi:10.1093/nar/15.4.1353}
\bibitem{bailey1994}
Timothy L. Bailey and Charles Elkan: ``Fitting a mixture model by expectation maximization to discover motifs in biopolymers'', \textit{Proceedings of the Second International Conference on Intelligent Systems for Molecular Biology} 28--36. AAAI Press, Menlo Park, California (1994).
\bibitem{chapman2000}
Brad Chapman and Jeff Chang: ``Biopython: Python tools for computational biology''. \textit{ACM SIGBIO Newsletter} {\bf 20} (2): 15--19 (August 2000).
\bibitem{dehoon2004}
Michiel J. L. de Hoon, Seiya Imoto, John Nolan, Satoru Miyano: ``Open source clustering software''. \textit{Bioinformatics} {\bf 20} (9): 1453--1454 (2004). \href{http://dx.doi.org/10.1093/bioinformatics/bth078}{doi:10.1093/bioinformatics/bth078}
\bibitem{eisen1998}
Michiel B. Eisen, Paul T. Spellman, Patrick O. Brown, David Botstein: ``Cluster analysis and display of genome-wide expression patterns''. \textit{Proceedings of the National Academy of Science USA} {\bf 95} (25): 14863--14868 (1998). \href{http://dx.doi.org/10.1073/pnas.96.19.10943-c}{doi:10.1073/pnas.96.19.10943-c}
\bibitem{golub1971}
Gene H. Golub, Christian Reinsch: ``Singular value decomposition and least squares solutions''. In \textit{Handbook for Automatic Computation}, {\bf 2}, (Linear Algebra) (J. H. Wilkinson and C. Reinsch, eds), 134--151. New York: Springer-Verlag (1971).
\bibitem{golub1989}
Gene H. Golub, Charles F. Van Loan: \textit{Matrix computations}, 2nd edition (1989).
\bibitem{hamelryck2003a}
Thomas Hamelryck and  Bernard Manderick: 11PDB parser and structure class
implemented in Python''. \textit{Bioinformatics}, \textbf{19} (17): 2308--2310 (2003) \href{http://dx.doi.org/10.1093/bioinformatics/btg299}{doi: 10.1093/bioinformatics/btg299}. 
\bibitem{hamelryck2003b}
Thomas Hamelryck: ``Efficient identification of side-chain patterns using a multidimensional index tree''. \textit{Proteins} {\bf 51} (1): 96--108 (2003). \href{http://dx.doi.org/10.1002/prot.10338}{doi:10.1002/prot.10338}
\bibitem{hamelryck2005}
Thomas Hamelryck: ``An amino acid has two sides; A new 2D measure provides a different view of solvent exposure''. \textit{Proteins} {\bf 59} (1): 29--48 (2005). \href{http://dx.doi.org/10.1002/prot.20379}{doi:10.1002/prot.20379}.
\bibitem{hartigan1975}
John A. Hartiga. \textit{Clustering algorithms}. New York: Wiley (1975).
\bibitem{jain1988}
Anil L. Jain, Richard C. Dubes: \textit{Algorithms for clustering data}. Englewood Cliffs, N.J.: Prentice Hall (1988).
\bibitem{kachitvichyanukul1988}
Voratas Kachitvichyanukul, Bruce W. Schmeiser: Binomial Random Variate Generation. \textit{Communications of the ACM} {\bf 31} (2): 216--222 (1988). \href{http://dx.doi.org/10.1145/42372.42381}{doi:10.1145/42372.42381}
\bibitem{kohonen1997}
Teuvo Kohonen: ``Self-organizing maps'', 2nd Edition. Berlin; New York: Springer-Verlag (1997).
\bibitem{lecuyer1988}
Pierre L'Ecuyer: ``Efficient and Portable Combined Random Number Generators.''
\textit{Communications of the ACM} {\bf 31} (6): 742--749,774 (1988). \href{http://dx.doi.org/10.1145/62959.62969}{doi:10.1145/62959.62969}
\bibitem{majumdar2005}
Indraneel Majumdar, S. Sri Krishna, Nick V. Grishin: ``PALSSE: A program to delineate linear secondary structural elements from protein structures.'' \textit{BMC Bioinformatics}, {\bf 6}: 202 (2005). \href{http://dx.doi.org/10.1186/1471-2105-6-202}{doi:10.1186/1471-2105-6-202}.
\bibitem{matys2003}
V. Matys, E. Fricke, R. Geffers, E. G\"ossling, M. Haubrock, R. Hehl, K. Hornischer, D. Karas, A.E. Kel, O.V. Kel-Margoulis, D.U. Kloos, S. Land, B. Lewicki-Potapov, H. Michael, R. M\"unch, I. Reuter, S. Rotert, H. Saxel, M. Scheer, S. Thiele, E. Wingender E: ``TRANSFAC: transcriptional regulation, from patterns to profiles.'' Nucleic Acids Research {\bf 31} (1): 374--378 (2003). \href{http://dx.doi.org/10.1093/nar/gkg108}{doi:10.1093/nar/gkg108}
\bibitem{sibson1973}
Robin Sibson: ``SLINK: An optimally efficient algorithm for the single-link cluster method''. \textit{The Computer Journal} {\bf 16} (1): 30--34 (1973). \href{http://dx.doi.org/10.1093/comjnl/16.1.30}{doi:10.1093/comjnl/16.1.30}
\bibitem{snedecor1989}
George W. Snedecor, William G. Cochran: \textit{Statistical methods}. Ames, Iowa: Iowa State University Press (1989).
\bibitem{tamayo1999}
Pablo Tamayo, Donna Slonim, Jill Mesirov, Qing Zhu, Sutisak Kitareewan, Ethan Dmitrovsky, Eric S. Lander, Todd R. Golub: ``Interpreting patterns of gene expression with self-organizing maps: Methods and application to hematopoietic differentiation''. \textit{Proceedings of the National Academy of Science USA} {\bf 96} (6): 2907--2912 (1999). \href{http://dx.doi.org/10.1073/pnas.96.6.2907}{doi:10.1073/pnas.96.6.2907}
\bibitem{tryon1970}
Robert C. Tryon, Daniel E. Bailey: \textit{Cluster analysis}. New York: McGraw-Hill (1970).
\bibitem{tukey1977}
John W. Tukey: ``Exploratory data analysis''. Reading, Mass.: Addison-Wesley Pub. Co. (1977).
\bibitem{yeung2001}
Ka Yee Yeung, Walter L. Ruzzo: ``Principal Component Analysis for clustering gene expression data''. \textit{Bioinformatics} {\bf 17} (9): 763--774 (2001). \href{http://dx.doi.org/10.1093/bioinformatics/17.9.763}{doi:10.1093/bioinformatics/17.9.763}
\bibitem{saldanha2004}
Alok Saldanha: ``Java Treeview---extensible visualization of microarray data''. \textit{Bioinformatics} {\bf 20} (17): 3246--3248 (2004). 
\href{http://dx.doi.org/10.1093/bioinformatics/bth349}{http://dx.doi.org/10.1093/bioinformatics/bth349}
\end{thebibliography}
