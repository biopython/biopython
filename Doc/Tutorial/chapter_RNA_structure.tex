\chapter{Bio.RNA_structure}

RNA_structure could become a Biopython module supporting RNA structure search, download,
prediction and comparison.

The medium term objective for the module is to support two internet databases
for RNA sequences and structures, offer 2D RNA structure prediction
and comparison of both 3D and 2D RNA structures.

\section{APIs for RNA databases}

As a first element of our package we have prepared APIs to access two RNA
databases. After presenting both of them, we will also introduce a unified
way to use them.

\subsection{API_NDB}

Nucleic Acid Database (\url{http://ndbserver.rutgers.edu/}) is the biggest
and the most well known database storing both DNA and RNA molecules and complexes.
The biggest advantage of the database beside number of stored structure
is its link with PDB, enabling easy access and unified ID type.

Therefore, we have decided to write an API to access the RNA part of the database,
search through it and download data. It is possible to search the database using
both PDB ID of a RNA molecule and part of a seqence. Second functionality
uses blasting to give user 3 best matches. After finding the desired RNA
it is possible to save its sequence in FASTA format, structure in PDB format
and metadata, including data about experiments to obtain structure, publication
details etc.

\begin{verbatim}
>>> from RNA_structure import API_NDB
>>> sample = Nucleic_scid_database(pdb_id = "5SWE")
>>> sample.download_fasta_sequence()
>>> sample.download_pdb_structure()
>>> sample.metadata_to_file()
\end{verbatim}

As you can see in the above example, you need to first define a class of your
request and afterwards use single functions to download the data.

\subsection{API_RNA_STRAND}

RNA Strand database (\url{http://www.rnasoft.ca/strand/}) is another
database storing RNA molecules. It contains known RNA secondary structures
of any type and organism gathering data from a lot of smaller 2D RNA structure
databases. Searching the database is rather simple, however data download
is not so straight forward. The downside of the database is a totally different
system of IDs created here de novo for all of the molecules.

Our API searches through the database using a sequence given as an argument.
It finds at most 100 similar molecules and enables user to choose the desired
RNA. Afterwards, easy functions let download of sequence, structure or metadata.
Sequence is saved in FASTA format, whereas for structure we have chosen
the BPSEQ format.

\begin{verbatim}
>>> from RNA_structure import API_RNA_STRAND
>>> sample = RNA_STRAND(sequence = 'UAAGCCCUA')
>>> sample.download_fasta_sequence()
>>> sample.download_pdb_structure()
>>> sample.metadata_to_file()
\end{verbatim}

As you can see in the above example, names of functions are the same as in the
API for Nucleic Acid Database.

\subsection{API_function}

The similarity of both API classes let us prepare the small class to unify
the usage of both databases. User needs to define the desired database,
recquired output and searched query. Additionally, user can specify the path,
where the output data will be saved.

In the below example we present a very siple use of our unified interface for both APIs.

\begin{verbatim}
>>> from RNA_structure import API_function
>>> x = RNA_API("ndb","sequence","ACCGUACG")
>>> x.use_API()
\end{verbatim}

\section{RNAfold_wrapper}

For RNA 2D structure prediction we have decided to use a part of very well known
Vienna Package (\url{http://www.tbi.univie.ac.at/RNA/}). It consists of multiple
tools, however to wrap it we have chosen only one part - RNAfold. It calculates minimum
free energy secondary structures and partition function of RNAs. The basic use of
this software is described on the webpage of Vienna University
(\url{http://www.tbi.univie.ac.at/RNA/}).

All the possible options provided for the RNAfold were wrapped using Python module
Bio.Application. Therefore, all the options usually inherit names from original ones.
The whole wrapper is fully documented with the docstrings describing single options.

Below in the example we present a basic usege of the wrapper.

\begin{verbatim}
>>> cline = RNAfoldCommandLine(infile="~/Desktop/TMR_00200_sequence.fasta",energyModel=0, dangles=2,maxBPspan=1)
>>> std_output, err_output = cline()
\end{verbatim}

The infile has to be always specified and all of the additional details are optional
and they don't have to be defined.

Wrapper itsel returns only the output which will be normally observed in command line.
However, we have written simple functions to return structure and free energy value.

\begin{verbatim}
>>> print(dot_parenthesis(std_output))
>>> print(energy(std_output))
\end{verbatim}

\section{dot2bpseq_converter}

According to very different formats of RNA secondary structures, we were searching for
a simple converter. However, there was nothing that stood out to our expectations.
Therefore we have decided to prepare a converter to change the output of RNAfold to
the file format that our secondary structure comparison will support. We have written
a script to convert dot-paretheses notation (which is typical for Vienna Package) to
bpseq format.

It is compatible with our output from RNAfold and it is a part of our pipline:
it takes the output from RNAfold wrapper and bypasses it to the RBP_score algorithm.

\section{RNA structures comparison}

Next part of our package is responsible for comparison of 3D and 2D RNA structures.

\subsection{RBP_score}

To compare RNA secondary structures we have implemented an algorithm - 
relaxed base pair score described in a publication from 2010
"Comparing RNA secondary structures using a relaxed base-pair score" written by
Phaedra Agius, Kristin P. Bennett, and Michael Zuker and published in RNA journal.
It is a very simple but powerful modification of previously described and widely used
base pair score. Base pair score was simply counting the base pairs which
were present in only one structure from two compared ones. Relaxed base pair score
needs one additional parameter (chosen by the algorithm user). It is named
relaxation parameter. According to this one the relaxed base pair score is calculated.

Our implementation is flexible and user can choose both - base pair score and
relaxed base pair score. In the second case program will ask about the desired
value of the relaxation parameter.

\subsection{RMSD}

RMSD which stands for root-mean-square deviation is the measure of average distance
between atoms and serves widely for the comparison of 3D RNA structures.

We have used an already written script found on GitHub (\url{https://github.com/charnley/rmsd})
written by Jimmy Charnley Kromann and Lars Andersen Bratholm
which takes two file paths and calculates RMSD between files present there.