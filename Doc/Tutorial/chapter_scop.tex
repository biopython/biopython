\chapter{Query the SCOP database with Bio.SCOP}
\label{chapter:scop}

SCOP (Structural Classification of Proteins) is a database that stores protein domains and 
the relationships between them. It uses several levels of organization:
\begin{itemize}
\item Class (CL)
\item Fold (CF)
\item Superfamily (SF)
\item Family (FA)
\item Protein (PR)
\item Protein species (SP)
\end{itemize}

Several variants of SCOP exist. The original SCOP project - with versions up to 1.75 - 
represents the relationships between domains using a tree structure, and was created 
largely by manual curation. Its successor, SCOPe (currently on version 2.07) extends 
it using automated tools. Most recently, the SCOP2 database has revised SCOP by using a 
directed acyclic graph instead of a tree.

Every node is given a 7-digit identifier, with the first digit represeting the level (so that 
the id of a protein species begins with 8, and the id of a class begins with 1). 

\section{The SCOPRemote object for remote access}
\label{sec:SCOPRemote}

\begin{minted}{pycon}
>>> from Bio.SCOP import SCOPRemote
>>> import pprint #for pretty-printing dictionaries
>>> scopConnection = SCOPRemote()
\end{minted}

To retrieve information about the database version, use the \verb|stats()| method.

\begin{minted}{pycon}
>>> stats = scopConnection.stats() #retrieve statistics as a dictionary
>>> pprint.pprint(stats) #print Python dictionary cleanly
{'counts': [['folds', 1398],
            ['IUPR', 18],
            ['hyperfamilies', 15],
            ['superfamilies', 2485],
            ['families', 5134],
            ['inter-relationships', 59]],
 'sources': {'PDB': '2020-01-03',
             'SIFTS': '2020-01-04',
             'UniProt': '2019-11-31'},
 'stats': {'domains': '41,104',
           'release_date': '2020-01-17',
           'structures': '504,937'}}
\end{minted}

\subsection{Information on one node}

The \verb|term()| method provides information about a node in the SCOP graph. Examples are taken from the Glypican family of proteins.

\begin{minted}{pycon}
>>> node_info = scopConnection.term('8048617')
>>> pprint.pprint(node_info)
{'counts': [['folds', 1398],
            ['IUPR', 18],
            ['hyperfamilies', 15],
            ['superfamilies', 2485],
            ['families', 5134],
            ['inter-relationships', 59]],
 'sources': {'PDB': '2020-01-03',
             'SIFTS': '2020-01-04',
             'UniProt': '2019-11-31'},
 'stats': {'domains': '41,104',
           'release_date': '2020-01-17',
           'structures': '504,937'}}
>>> node_info = scopConnection.term('8048617')
>>> pprint.pprint(node_info)
{'children': [],
 'description': 'Glypican-1',
 'id': 8048617,
 'name': '4ACR D:29-475',
 'pdb_begin': '29',
 'pdb_chain': 'D',
 'pdb_end': '475',
 'pdb_id': '4ACR',
 'pdb_segments': [['D', 29, 475]],
 'protein_segments': [[29, 475]],
 'protein_species': 'Homo sapiens',
 'rank': 8,
 'repre_seq': 16820,
 'seq_begin': '29',
 'seq_end': '475',
 'sequence': 'MELRARGWWLLCAAAALVACARGDPASKSRSCGEVRQIYGAKGFSLSDVPQAEISGEH ... LALTVARPRWR',
 'type': 'domain',
 'uniprot_id': 'P35052'}
\end{minted}

\subsection{A node's ancestors}

The \verb|ancestry()| method shows the ancestors of a node.

\begin{minted}{pycon}
>>> ancestry = scopConnection.ancestry('4006165')
>>> pprint.pprint(ancestry)
{'id': 4006165,
 'lineage': {'edges': [[4006165, 3001950, 'is'],
                       [3001950, 2001193, 'is'],
                       [2001193, 1000000, 'is']],
             'nodes': {'1000000': {'id': 1000000,
                                   'name': 'All alpha proteins',
                                   'type': 'class'},
                       '2001193': {'id': 2001193,
                                   'name': 'Frizzled domain-like',
                                   'type': 'fold'},
                       '3001950': {'id': 3001950,
                                   'name': 'Frizzled domain-like',
                                   'type': 'superfamily'},
                       '4006165': {'id': 4006165,
                                   'name': 'Glypican-like',
                                   'type': 'family'}}}}
\end{minted}

\subsection{A node's (leaf) descendants}

The \verb|domains()| method shows the graph "leaves" that descend from the node.

\begin{minted}{pycon}
>>> domains = scopConnection.domains('4006165')
>>> pprint.pprint(domains)
{'domains': [{'id': 8048617,
              'num': 13,
              'pdb_id': '4ACR',
              'pdb_regions': [['D', 29, 475]],
              'protein_name': 'Glypican-1',
              'protein_regions': [[29, 475]],
              'species': 'Homo sapiens',
              'type': 'FA',
              'uniprot_id': 'P35052'},
             {'id': 8048618,
              'num': 0,
              'pdb_id': '3ODN',
              'pdb_regions': [['A', 117, 614]],
              'protein_name': 'Dally-like, isoform ALD47466p',
              'protein_regions': [[117, 614]],
              'species': 'Drosophila melanogaster',
              'type': 'FA',
              'uniprot_id': 'Q9VUG1'}],
 'id': 4006165}
\end{minted}

\section{The Scop object for local access}
\label{sec:ScopObject}

The Scop object can be used for constructing a local copy of the SCOP database from a file or from other sources.
