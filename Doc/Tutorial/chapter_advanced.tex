\chapter{Advanced}
\label{chapter:advanced}

\section{Parser Design}

Many of the older Biopython parsers were built around an event-oriented
design that includes Scanner and Consumer objects.

Scanners take input from a data source and analyze it line by line,
sending off an event whenever it recognizes some information in the
data.  For example, if the data includes information about an organism
name, the scanner may generate an \verb|organism_name| event whenever it
encounters a line containing the name.

Consumers are objects that receive the events generated by Scanners.
Following the previous example, the consumer receives the
\verb|organism_name| event, and the processes it in whatever manner
necessary in the current application.

This is a very flexible framework, which is advantageous if you want to
be able to parse a file format into more than one representation.  For
example, the \verb|Bio.GenBank| module uses this to construct either
\verb|SeqRecord| objects or file-format-specific record objects.

More recently, many of the parsers added for \verb|Bio.SeqIO| and
\verb|Bio.AlignIO| take a much simpler approach, but only generate a
single object representation (\verb|SeqRecord| and
\verb|MultipleSeqAlignment| objects respectively). In some cases the
\verb|Bio.SeqIO| parsers actually wrap
another Biopython parser - for example, the \verb|Bio.SwissProt| parser
produces SwissProt format specific record objects, which get converted
into \verb|SeqRecord| objects.
