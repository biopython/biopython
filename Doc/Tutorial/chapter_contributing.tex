\chapter{Where to go from here -- contributing to Biopython}

\section{Bug Reports + Feature Requests}

Getting feedback on the Biopython modules is very important to us. Open-source projects like this benefit greatly from feedback, bug-reports (and patches!) from a wide variety of contributors.

The main forum for discussing feature requests and potential bugs is the
\href{http://biopython.org/wiki/Mailing_lists}{Biopython mailing list}:

\begin{itemize}
  \item \mailto{biopython@biopython.org} -- An unmoderated list for discussion of anything to do with Biopython.
\end{itemize}

Additionally, if you think you've found a new bug, you can submit it to
our issue tracker at \url{https://github.com/biopython/biopython/issues}
(this replaced the older Open Bioinformatics Foundation hosted RedMine tracker).
This way, it won't get buried in anyone's Inbox and forgotten about.

\section{Mailing lists and helping newcomers}

We encourage all our uses to sign up to the main Biopython mailing list.
Once you've got the hang of an area of Biopython, we'd encourage you to
help answer questions from beginners.  After all, you were a beginner once.

\section{Contributing Documentation}

We're happy to take feedback or contributions - either via a bug-report or on the Mailing List.
While reading this tutorial, perhaps you noticed some topics you were interested in which were missing, or not clearly explained.  There is also Biopython's built in documentation (the docstrings, these are also
\href{http://biopython.org/DIST/docs/api}{online}), where again, you may be able to help fill in any blanks.

\section{Contributing cookbook examples}
As explained in Chapter~\ref{chapter:cookbook}, Biopython now has a wiki
collection of user contributed ``cookbook'' examples,
\url{http://biopython.org/wiki/Category:Cookbook} -- maybe you can add
to this?

\section{Maintaining a distribution for a platform}
\label{sec:maintain_dist}

We currently provide source code archives (suitable for any OS, if you have the right build tools installed), and Windows Installers which are just click and run.  This covers all the major operating systems.

Most major Linux distributions have volunteers who take these source code releases, and compile them into packages for Linux users to easily install (taking care of dependencies etc).  This is really great and we are of course very grateful.  If you would like to contribute to this work, please find out more about how your Linux distribution handles this.

Below are some tips for certain platforms to maybe get people started with helping out:

\begin{description}

\item[Windows] -- Windows products typically have a nice graphical installer that installs all of the essential components in the right place. We use Distutils to create a installer of this type fairly easily.

You must first make sure you have a C compiler on your Windows computer, and that you can compile and install things (this is the hard bit - see the Biopython installation instructions for info on how to do this).

Once you are setup with a C compiler, making the installer just requires doing:

\begin{verbatim}
python setup.py bdist_wininst
\end{verbatim}

Now you've got a Windows installer. Congrats!  At the moment we have no trouble shipping installers built on 32 bit windows.  If anyone would like to look into supporting 64 bit Windows that would be great.

  \item[RPMs] -- RPMs are pretty popular package systems on some Linux platforms. There is lots of documentation on RPMs available at \url{http://www.rpm.org} to help you get started with them. To create an RPM for your platform is really easy. You just need to be able to build the package from source (having a C compiler that works is thus essential) -- see the Biopython installation instructions for more info on this.

To make the RPM, you just need to do:

\begin{verbatim}
python setup.py bdist_rpm
\end{verbatim}

This will create an RPM for your specific platform and a source RPM in the directory \verb|dist|. This RPM should be good and ready to go, so this is all you need to do! Nice and easy.

\item[Macintosh] -- Since Apple moved to Mac OS X, things have become much easier on the Mac.  We generally
treat it as just another Unix variant, and installing Biopython from source is just as easy as on Linux.
The easiest way to get all the GCC compilers etc installed is to install Apple's X-Code.
We might be able to provide click and run installers for Mac OS X, but to date there hasn't been any demand.

\end{description}

Once you've got a package, please test it on your system to make sure it installs everything in a good way and seems to work properly. Once you feel good about it, send it off to one of the Biopython developers (write to our main mailing list at biopython@biopython.org if you're not sure who to send it to) and you've done it. Thanks!

\section{Contributing Unit Tests}

Even if you don't have any new functionality to add to Biopython, but you want to write some code, please
consider extending our unit test coverage.  We've devoted all of Chapter~\ref{sec:regr_test} to this topic.

\section{Contributing Code}

There are no barriers to joining Biopython code development other
than an interest in creating biology-related code in Python. The
best place to express an interest is on the Biopython mailing lists
-- just let us know you are interested in coding and what kind of
stuff you want to work on. Normally, we try to have some discussion
on modules before coding them, since that helps generate good ideas
-- then just feel free to jump right in and start coding!

The main Biopython release tries to be fairly uniform and interworkable,
to make it easier for users. You can read about some of (fairly
informal) coding style guidelines we try to use in Biopython in the
contributing documentation at
\url{http://biopython.org/wiki/Contributing}. We also try to add code to the distribution along with tests (see Chapter~\ref{sec:regr_test} for more info on the regression testing framework) and documentation, so that everything can stay as workable and well documented as possible (including docstrings). This is, of course, the most ideal situation, under many situations you'll be able to find other people on the list who will be willing to help add documentation or more tests for your code once you make it available. So, to end this paragraph like the last, feel free to start working!

Please note that to make a code contribution you must have the legal right to contribute it and license it under the Biopython license.  If you wrote it all yourself, and it is not based on any other code, this shouldn't be a problem.  However, there are issues if you want to contribute a derivative work - for example something based on GPL or LPGL licenced code would not be compatible with our license.  If you have any queries on this, please discuss the issue on the mailing list or GitHub.

Another point of concern for any additions to Biopython regards any build time or run time dependencies.  Generally speaking, writing code to interact with a standalone tool (like BLAST, EMBOSS or ClustalW) doesn't present a big problem.  However, any dependency on another library - even a Python library (especially one needed in order to compile and install Biopython like NumPy) would need further discussion.

Additionally, if you have code that you don't think fits in the
distribution, but that you want to make available, we maintain Script
Central (\url{http://biopython.org/wiki/Scriptcentral})
which has pointers to freely available code in Python for bioinformatics.

Hopefully this documentation has got you excited enough about
Biopython to try it out (and most importantly, contribute!). Thanks
for reading all the way through!

