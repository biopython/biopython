% This is the main LaTeX file which is used to produce the Biopython
% Tutorial documentation.
%
% If you just want to read the documentation, you can pick up ready-to-go
% copies in both pdf and html format from:
%
% http://biopython.org/DIST/docs/tutorial/Tutorial.html
% http://biopython.org/DIST/docs/tutorial/Tutorial.pdf
%
% If you want to typeset the documentation, you'll need a standard TeX/LaTeX
% distribution (I use teTeX, which works great for me on Unix platforms).
% Additionally, you need HeVeA (or at least hevea.sty), which can be
% found at:
%
% http://pauillac.inria.fr/~maranget/hevea/index.html
%
% You will also need the pictures included in the document, some of
% which are UMLish diagrams created by Dia
% (http://www.lysator.liu.se/~alla/dia/dia.html).
% These diagrams are available from Biopython git in the original dia
% format, which you can easily save as .png format using Dia itself.
% They are also checked in as the png files, so if you make
% modifications to the original dia files, the png files should also be
% changed.
%
% Once you're all set, you should be able to generate pdf by running:
%
% pdflatex Tutorial.tex  (to generate the first draft)
% pdflatex Tutorial.tex  (to get the cross references right)
% pdflatex Tutorial.tex  (to get the table of contents right)
%
% To generate the html, you'll need HeVeA installed. You should be
% able to just run:
%
% hevea -fix Tutorial.tex
%
% However, on older versions of hevea you may first need to remove the
% Tutorial.aux file generated by LaTeX, then run hevea twice to get
% the references right.
%
% If you want to typeset this and have problems, please report them
% at biopython-dev@biopython.org, and we'll try to get things resolved. We
% always love to have people interested in the documentation!

\documentclass{report}
\usepackage{url}
\usepackage{fullpage}
\usepackage{hevea}
\usepackage{graphicx}

% make everything have section numbers
\setcounter{secnumdepth}{4}

% Make links between references
\usepackage{hyperref}
\newif\ifpdf
\ifx\pdfoutput\undefined
  \pdffalse
\else
  \pdfoutput=1
  \pdftrue
\fi
\ifpdf
  \hypersetup{colorlinks=true, hyperindex=true, citecolor=red, urlcolor=blue}
\fi

\begin{document}

\begin{htmlonly}
\title{Biopython Tutorial and Cookbook}
\end{htmlonly}
\begin{latexonly}
\title{
%Hack to get the logo on the PDF front page:
\includegraphics[width=\textwidth]{images/biopython.jpg}\\
%Hack to get some white space using a blank line:
~\\
Biopython Tutorial and Cookbook}
\end{latexonly}

\author{Jeff Chang, Brad Chapman, Iddo Friedberg, Thomas Hamelryck, \\
Michiel de Hoon, Peter Cock, Tiago Antao, Eric Talevich, Bartek Wilczy\'{n}ski}
\date{Last Update -- 16 February 2014 (Biopython 1.63+)}

%Hack to get the logo at the start of the HTML front page:
%(hopefully this isn't going to be too wide for most people)
\begin{rawhtml}
<P ALIGN="center">
<IMG ALIGN="center" SRC="images/biopython.jpg" TITLE="Biopython Logo" ALT="[Biopython Logo]" width="1024" height="288" />
</p>
\end{rawhtml}

\maketitle
\tableofcontents

% chapter1 Introduction
\include{chapter/chapter1}

% chapter2 Quick Start -- What can you do with Biopython?
\include{chapter/chapter2}

% chapter3 Sequence objects
\include{chapter/chapter3}

% chapter4 Sequence annotation objects
\include{chapter/chapter4}

% chapter5 Sequence Input/Output
\include{chapter/chapter5}

% chapter6 Multiple Sequence Alignment objects
\chapter{Multiple Sequence Alignment objects}
\label{chapter:Bio.AlignIO}

This chapter is about Multiple Sequence Alignments, by which we mean a collection of
multiple sequences which have been aligned together -- usually with the insertion of gap
characters, and addition of leading or trailing gaps -- such that all the sequence
strings are the same length. Such an alignment can be regarded as a matrix of letters,
where each row is held as a \verb|SeqRecord| object internally.

We will introduce the \verb|MultipleSeqAlignment| object which holds this kind of data,
and the \verb|Bio.AlignIO| module for reading and writing them as various file formats
(following the design of the \verb|Bio.SeqIO| module from the previous chapter).
Note that both \verb|Bio.SeqIO| and \verb|Bio.AlignIO| can read and write sequence
alignment files.  The appropriate choice will depend largely on what you want to do
with the data.

The final part of this chapter is about our command line wrappers for common multiple
sequence alignment tools like ClustalW and MUSCLE.

\section{Parsing or Reading Sequence Alignments}

We have two functions for reading in sequence alignments, \verb|Bio.AlignIO.read()| and \verb|Bio.AlignIO.parse()| which following the convention introduced in \verb|Bio.SeqIO| are for files containing one or multiple alignments respectively.

Using \verb|Bio.AlignIO.parse()| will return an {\it iterator} which gives \verb|MultipleSeqAlignment| objects.  Iterators are typically used in a for loop.  Examples of situations where you will have multiple different alignments include resampled alignments from the PHYLIP tool \verb|seqboot|, or multiple pairwise alignments from the EMBOSS tools \verb|water| or \verb|needle|, or Bill Pearson's FASTA tools.

However, in many situations you will be dealing with files which contain only a single alignment.  In this case, you should use the \verb|Bio.AlignIO.read()| function which returns a single \verb|MultipleSeqAlignment| object.

Both functions expect two mandatory arguments:

\begin{enumerate}
\item The first argument is a {\it handle} to read the data from, typically an open file (see Section~\ref{sec:appendix-handles}), or a filename.
\item The second argument is a lower case string specifying the alignment format.  As in \verb|Bio.SeqIO| we don't try and guess the file format for you!  See \url{http://biopython.org/wiki/AlignIO} for a full listing of supported formats.
\end{enumerate}

\noindent There is also an optional \verb|seq_count| argument which is discussed in Section~\ref{sec:AlignIO-count-argument} below for dealing with ambiguous file formats which may contain more than one alignment.

A further optional \verb|alphabet| argument allowing you to specify the expected alphabet. This can be useful as many alignment file formats do not explicitly label the sequences as RNA, DNA or protein -- which means \verb|Bio.AlignIO| will default to using a generic alphabet.

\subsection{Single Alignments}
As an example, consider the following annotation rich protein alignment in the PFAM or Stockholm file format:

\begin{verbatim}
# STOCKHOLM 1.0
#=GS COATB_BPIKE/30-81  AC P03620.1
#=GS COATB_BPIKE/30-81  DR PDB; 1ifl ; 1-52;
#=GS Q9T0Q8_BPIKE/1-52  AC Q9T0Q8.1
#=GS COATB_BPI22/32-83  AC P15416.1
#=GS COATB_BPM13/24-72  AC P69541.1
#=GS COATB_BPM13/24-72  DR PDB; 2cpb ; 1-49;
#=GS COATB_BPM13/24-72  DR PDB; 2cps ; 1-49;
#=GS COATB_BPZJ2/1-49   AC P03618.1
#=GS Q9T0Q9_BPFD/1-49   AC Q9T0Q9.1
#=GS Q9T0Q9_BPFD/1-49   DR PDB; 1nh4 A; 1-49;
#=GS COATB_BPIF1/22-73  AC P03619.2
#=GS COATB_BPIF1/22-73  DR PDB; 1ifk ; 1-50;
COATB_BPIKE/30-81             AEPNAATNYATEAMDSLKTQAIDLISQTWPVVTTVVVAGLVIRLFKKFSSKA
#=GR COATB_BPIKE/30-81  SS    -HHHHHHHHHHHHHH--HHHHHHHH--HHHHHHHHHHHHHHHHHHHHH----
Q9T0Q8_BPIKE/1-52             AEPNAATNYATEAMDSLKTQAIDLISQTWPVVTTVVVAGLVIKLFKKFVSRA
COATB_BPI22/32-83             DGTSTATSYATEAMNSLKTQATDLIDQTWPVVTSVAVAGLAIRLFKKFSSKA
COATB_BPM13/24-72             AEGDDP...AKAAFNSLQASATEYIGYAWAMVVVIVGATIGIKLFKKFTSKA
#=GR COATB_BPM13/24-72  SS    ---S-T...CHCHHHHCCCCTCCCTTCHHHHHHHHHHHHHHHHHHHHCTT--
COATB_BPZJ2/1-49              AEGDDP...AKAAFDSLQASATEYIGYAWAMVVVIVGATIGIKLFKKFASKA
Q9T0Q9_BPFD/1-49              AEGDDP...AKAAFDSLQASATEYIGYAWAMVVVIVGATIGIKLFKKFTSKA
#=GR Q9T0Q9_BPFD/1-49   SS    ------...-HHHHHHHHHHHHHHHHHHHHHHHHHHHHHHHHHHHHHHHH--
COATB_BPIF1/22-73             FAADDATSQAKAAFDSLTAQATEMSGYAWALVVLVVGATVGIKLFKKFVSRA
#=GR COATB_BPIF1/22-73  SS    XX-HHHH--HHHHHH--HHHHHHH--HHHHHHHHHHHHHHHHHHHHHHH---
#=GC SS_cons                  XHHHHHHHHHHHHHHHCHHHHHHHHCHHHHHHHHHHHHHHHHHHHHHHHC--
#=GC seq_cons                 AEssss...AptAhDSLpspAT-hIu.sWshVsslVsAsluIKLFKKFsSKA
//
\end{verbatim}

This is the seed alignment for the Phage\_Coat\_Gp8 (PF05371) PFAM entry, downloaded from a now out of date release of PFAM from \url{http://pfam.sanger.ac.uk/}.  We can load this file as follows (assuming it has been saved to disk as ``PF05371\_seed.sth'' in the current working directory):

%doctest examples
\begin{verbatim}
>>> from Bio import AlignIO
>>> alignment = AlignIO.read("PF05371_seed.sth", "stockholm")
\end{verbatim}

\noindent This code will print out a summary of the alignment:

%cont-doctest
\begin{verbatim}
>>> print(alignment)
SingleLetterAlphabet() alignment with 7 rows and 52 columns
AEPNAATNYATEAMDSLKTQAIDLISQTWPVVTTVVVAGLVIRL...SKA COATB_BPIKE/30-81
AEPNAATNYATEAMDSLKTQAIDLISQTWPVVTTVVVAGLVIKL...SRA Q9T0Q8_BPIKE/1-52
DGTSTATSYATEAMNSLKTQATDLIDQTWPVVTSVAVAGLAIRL...SKA COATB_BPI22/32-83
AEGDDP---AKAAFNSLQASATEYIGYAWAMVVVIVGATIGIKL...SKA COATB_BPM13/24-72
AEGDDP---AKAAFDSLQASATEYIGYAWAMVVVIVGATIGIKL...SKA COATB_BPZJ2/1-49
AEGDDP---AKAAFDSLQASATEYIGYAWAMVVVIVGATIGIKL...SKA Q9T0Q9_BPFD/1-49
FAADDATSQAKAAFDSLTAQATEMSGYAWALVVLVVGATVGIKL...SRA COATB_BPIF1/22-73
\end{verbatim}

You'll notice in the above output the sequences have been truncated.  We could instead write our own code to format this as we please by iterating over the rows as \verb|SeqRecord| objects:

%doctest examples
\begin{verbatim}
>>> from Bio import AlignIO
>>> alignment = AlignIO.read("PF05371_seed.sth", "stockholm")
>>> print("Alignment length %i" % alignment.get_alignment_length())
Alignment length 52
>>> for record in alignment:
...     print("%s - %s" % (record.seq, record.id))
AEPNAATNYATEAMDSLKTQAIDLISQTWPVVTTVVVAGLVIRLFKKFSSKA - COATB_BPIKE/30-81
AEPNAATNYATEAMDSLKTQAIDLISQTWPVVTTVVVAGLVIKLFKKFVSRA - Q9T0Q8_BPIKE/1-52
DGTSTATSYATEAMNSLKTQATDLIDQTWPVVTSVAVAGLAIRLFKKFSSKA - COATB_BPI22/32-83
AEGDDP---AKAAFNSLQASATEYIGYAWAMVVVIVGATIGIKLFKKFTSKA - COATB_BPM13/24-72
AEGDDP---AKAAFDSLQASATEYIGYAWAMVVVIVGATIGIKLFKKFASKA - COATB_BPZJ2/1-49
AEGDDP---AKAAFDSLQASATEYIGYAWAMVVVIVGATIGIKLFKKFTSKA - Q9T0Q9_BPFD/1-49
FAADDATSQAKAAFDSLTAQATEMSGYAWALVVLVVGATVGIKLFKKFVSRA - COATB_BPIF1/22-73
\end{verbatim}

You could also use the alignment object's \verb|format| method to show it in a particular file format  -- see Section~\ref{sec:alignment-format-method} for details.

Did you notice in the raw file above that several of the sequences include database cross-references to the PDB and the associated known secondary structure?  Try this:

%cont-doctest
\begin{verbatim}
>>> for record in alignment:
...     if record.dbxrefs:
...         print("%s %s" % (record.id, record.dbxrefs))
COATB_BPIKE/30-81 ['PDB; 1ifl ; 1-52;']
COATB_BPM13/24-72 ['PDB; 2cpb ; 1-49;', 'PDB; 2cps ; 1-49;']
Q9T0Q9_BPFD/1-49 ['PDB; 1nh4 A; 1-49;']
COATB_BPIF1/22-73 ['PDB; 1ifk ; 1-50;']
\end{verbatim}

\noindent To have a look at all the sequence annotation, try this:

\begin{verbatim}
>>> for record in alignment:
...     print(record)
\end{verbatim}

Sanger provide a nice web interface at \url{http://pfam.sanger.ac.uk/family?acc=PF05371} which will actually let you download this alignment in several other formats.  This is what the file looks like in the FASTA file format:

\begin{verbatim}
>COATB_BPIKE/30-81
AEPNAATNYATEAMDSLKTQAIDLISQTWPVVTTVVVAGLVIRLFKKFSSKA
>Q9T0Q8_BPIKE/1-52
AEPNAATNYATEAMDSLKTQAIDLISQTWPVVTTVVVAGLVIKLFKKFVSRA
>COATB_BPI22/32-83
DGTSTATSYATEAMNSLKTQATDLIDQTWPVVTSVAVAGLAIRLFKKFSSKA
>COATB_BPM13/24-72
AEGDDP---AKAAFNSLQASATEYIGYAWAMVVVIVGATIGIKLFKKFTSKA
>COATB_BPZJ2/1-49
AEGDDP---AKAAFDSLQASATEYIGYAWAMVVVIVGATIGIKLFKKFASKA
>Q9T0Q9_BPFD/1-49
AEGDDP---AKAAFDSLQASATEYIGYAWAMVVVIVGATIGIKLFKKFTSKA
>COATB_BPIF1/22-73
FAADDATSQAKAAFDSLTAQATEMSGYAWALVVLVVGATVGIKLFKKFVSRA
\end{verbatim}

\noindent Note the website should have an option about showing gaps as periods (dots) or dashes, we've shown dashes above.  Assuming you download and save this as file ``PF05371\_seed.faa'' then you can load it with almost exactly the same code:

\begin{verbatim}
from Bio import AlignIO
alignment = AlignIO.read("PF05371_seed.faa", "fasta")
print(alignment)
\end{verbatim}

All that has changed in this code is the filename and the format string.  You'll get the same output as before, the sequences and record identifiers are the same.
However, as you should expect, if you check each \verb|SeqRecord| there is no annotation nor database cross-references because these are not included in the FASTA file format.

Note that rather than using the Sanger website, you could have used \verb|Bio.AlignIO| to convert the original Stockholm format file into a FASTA file yourself (see below).

With any supported file format, you can load an alignment in exactly the same way just by changing the format string.  For example, use ``phylip'' for PHYLIP files, ``nexus'' for NEXUS files or ``emboss'' for the alignments output by the EMBOSS tools.  There is a full listing on the wiki page (\url{http://biopython.org/wiki/AlignIO}) and in the built in documentation (also \href{http://biopython.org/DIST/docs/api/Bio.AlignIO-module.html}{online}):

\begin{verbatim}
>>> from Bio import AlignIO
>>> help(AlignIO)
...
\end{verbatim}

\subsection{Multiple Alignments}

The previous section focused on reading files containing a single alignment.  In general however, files can contain more than one alignment, and to read these files we must use the \verb|Bio.AlignIO.parse()| function.

Suppose you have a small alignment in PHYLIP format:

\begin{verbatim}
    5    6
Alpha     AACAAC
Beta      AACCCC
Gamma     ACCAAC
Delta     CCACCA
Epsilon   CCAAAC
\end{verbatim}

If you wanted to bootstrap a phylogenetic tree using the PHYLIP tools, one of the steps would be to create a set of many resampled alignments using the tool \verb|bootseq|.  This would give output something like this, which has been abbreviated for conciseness:

\begin{verbatim}
    5     6
Alpha     AAACCA
Beta      AAACCC
Gamma     ACCCCA
Delta     CCCAAC
Epsilon   CCCAAA
    5     6
Alpha     AAACAA
Beta      AAACCC
Gamma     ACCCAA
Delta     CCCACC
Epsilon   CCCAAA
    5     6
Alpha     AAAAAC
Beta      AAACCC
Gamma     AACAAC
Delta     CCCCCA
Epsilon   CCCAAC
...
    5     6
Alpha     AAAACC
Beta      ACCCCC
Gamma     AAAACC
Delta     CCCCAA
Epsilon   CAAACC
\end{verbatim}

If you wanted to read this in using \verb|Bio.AlignIO| you could use:

%TODO - Replace the print blank line with print()?
\begin{verbatim}
from Bio import AlignIO
alignments = AlignIO.parse("resampled.phy", "phylip")
for alignment in alignments:
    print(alignment)
    print("")
\end{verbatim}

\noindent This would give the following output, again abbreviated for display:

\begin{verbatim}
SingleLetterAlphabet() alignment with 5 rows and 6 columns
AAACCA Alpha
AAACCC Beta
ACCCCA Gamma
CCCAAC Delta
CCCAAA Epsilon

SingleLetterAlphabet() alignment with 5 rows and 6 columns
AAACAA Alpha
AAACCC Beta
ACCCAA Gamma
CCCACC Delta
CCCAAA Epsilon

SingleLetterAlphabet() alignment with 5 rows and 6 columns
AAAAAC Alpha
AAACCC Beta
AACAAC Gamma
CCCCCA Delta
CCCAAC Epsilon

...

SingleLetterAlphabet() alignment with 5 rows and 6 columns
AAAACC Alpha
ACCCCC Beta
AAAACC Gamma
CCCCAA Delta
CAAACC Epsilon
\end{verbatim}

As with the function \verb|Bio.SeqIO.parse()|, using \verb|Bio.AlignIO.parse()| returns an iterator.
If you want to keep all the alignments in memory at once, which will allow you to access them in any order, then turn the iterator into a list:

\begin{verbatim}
from Bio import AlignIO
alignments = list(AlignIO.parse("resampled.phy", "phylip"))
last_align = alignments[-1]
first_align = alignments[0]
\end{verbatim}

\subsection{Ambiguous Alignments}
\label{sec:AlignIO-count-argument}
Many alignment file formats can explicitly store more than one alignment, and the division between each alignment is clear.  However, when a general sequence file format has been used there is no such block structure.  The most common such situation is when alignments have been saved in the FASTA file format.  For example consider the following:

\begin{verbatim}
>Alpha
ACTACGACTAGCTCAG--G
>Beta
ACTACCGCTAGCTCAGAAG
>Gamma
ACTACGGCTAGCACAGAAG
>Alpha
ACTACGACTAGCTCAGG--
>Beta
ACTACCGCTAGCTCAGAAG
>Gamma
ACTACGGCTAGCACAGAAG
\end{verbatim}

\noindent This could be a single alignment containing six sequences (with repeated identifiers).  Or, judging from the identifiers, this is probably two different alignments each with three sequences, which happen to all have the same length.

What about this next example?

\begin{verbatim}
>Alpha
ACTACGACTAGCTCAG--G
>Beta
ACTACCGCTAGCTCAGAAG
>Alpha
ACTACGACTAGCTCAGG--
>Gamma
ACTACGGCTAGCACAGAAG
>Alpha
ACTACGACTAGCTCAGG--
>Delta
ACTACGGCTAGCACAGAAG
\end{verbatim}

\noindent Again, this could be a single alignment with six sequences.  However this time based on the identifiers we might guess this is three pairwise alignments which by chance have all got the same lengths.

This final example is similar:

\begin{verbatim}
>Alpha
ACTACGACTAGCTCAG--G
>XXX
ACTACCGCTAGCTCAGAAG
>Alpha
ACTACGACTAGCTCAGG
>YYY
ACTACGGCAAGCACAGG
>Alpha
--ACTACGAC--TAGCTCAGG
>ZZZ
GGACTACGACAATAGCTCAGG
\end{verbatim}

\noindent In this third example, because of the differing lengths, this cannot be treated as a single alignment containing all six records.  However, it could be three pairwise alignments.

Clearly trying to store more than one alignment in a FASTA file is not ideal.  However, if you are forced to deal with these as input files \verb|Bio.AlignIO| can cope with the most common situation where all the alignments have the same number of records.
One example of this is a collection of pairwise alignments, which can be produced by the EMBOSS tools \verb|needle| and \verb|water| -- although in this situation, \verb|Bio.AlignIO| should be able to understand their native output using ``emboss'' as the format string.

To interpret these FASTA examples as several separate alignments, we can use \verb|Bio.AlignIO.parse()| with the optional \verb|seq_count| argument which specifies how many sequences are expected in each alignment (in these examples, 3, 2 and 2 respectively).
For example, using the third example as the input data:

%TODO - Replace the print blank line with print()?
\begin{verbatim}
for alignment in AlignIO.parse(handle, "fasta", seq_count=2):
    print("Alignment length %i" % alignment.get_alignment_length())
    for record in alignment:
        print("%s - %s" % (record.seq, record.id))
    print("")
\end{verbatim}

\noindent giving:

\begin{verbatim}
Alignment length 19
ACTACGACTAGCTCAG--G - Alpha
ACTACCGCTAGCTCAGAAG - XXX

Alignment length 17
ACTACGACTAGCTCAGG - Alpha
ACTACGGCAAGCACAGG - YYY

Alignment length 21
--ACTACGAC--TAGCTCAGG - Alpha
GGACTACGACAATAGCTCAGG - ZZZ
\end{verbatim}

Using \verb|Bio.AlignIO.read()| or \verb|Bio.AlignIO.parse()| without the \verb|seq_count| argument would give a single alignment containing all six records for the first two examples.  For the third example, an exception would be raised because the lengths differ preventing them being turned into a single alignment.

If the file format itself has a block structure allowing \verb|Bio.AlignIO| to determine the number of sequences in each alignment directly, then the \verb|seq_count| argument is not needed.  If it is supplied, and doesn't agree with the file contents, an error is raised.

Note that this optional \verb|seq_count| argument assumes each alignment in the file has the same number of sequences.  Hypothetically you may come across stranger situations, for example a FASTA file containing several alignments each with a different number of sequences -- although I would love to hear of a real world example of this.  Assuming you cannot get the data in a nicer file format, there is no straight forward way to deal with this using \verb|Bio.AlignIO|.  In this case, you could consider reading in the sequences themselves using \verb|Bio.SeqIO| and batching them together to create the alignments as appropriate.

\section{Writing Alignments}

We've talked about using \verb|Bio.AlignIO.read()| and \verb|Bio.AlignIO.parse()| for alignment input (reading files), and now we'll look at \verb|Bio.AlignIO.write()| which is for alignment output (writing files).  This is a function taking three arguments: some \verb|MultipleSeqAlignment| objects (or for backwards compatibility the obsolete \verb|Alignment| objects), a handle or filename to write to, and a sequence format.

Here is an example, where we start by creating a few \verb|MultipleSeqAlignment| objects the hard way (by hand, rather than by loading them from a file).
Note we create some \verb|SeqRecord| objects to construct the alignment from.

\begin{verbatim}
from Bio.Alphabet import generic_dna
from Bio.Seq import Seq
from Bio.SeqRecord import SeqRecord
from Bio.Align import MultipleSeqAlignment

align1 = MultipleSeqAlignment([
             SeqRecord(Seq("ACTGCTAGCTAG", generic_dna), id="Alpha"),
             SeqRecord(Seq("ACT-CTAGCTAG", generic_dna), id="Beta"),
             SeqRecord(Seq("ACTGCTAGDTAG", generic_dna), id="Gamma"),
         ])

align2 = MultipleSeqAlignment([
             SeqRecord(Seq("GTCAGC-AG", generic_dna), id="Delta"),
             SeqRecord(Seq("GACAGCTAG", generic_dna), id="Epsilon"),
             SeqRecord(Seq("GTCAGCTAG", generic_dna), id="Zeta"),
         ])

align3 = MultipleSeqAlignment([
             SeqRecord(Seq("ACTAGTACAGCTG", generic_dna), id="Eta"),
             SeqRecord(Seq("ACTAGTACAGCT-", generic_dna), id="Theta"),
             SeqRecord(Seq("-CTACTACAGGTG", generic_dna), id="Iota"),
         ])

my_alignments = [align1, align2, align3]
\end{verbatim}

\noindent Now we have a list of \verb|Alignment| objects, we'll write them to a PHYLIP format file:

\begin{verbatim}
from Bio import AlignIO
AlignIO.write(my_alignments, "my_example.phy", "phylip")
\end{verbatim}

\noindent And if you open this file in your favourite text editor it should look like this:

\begin{verbatim}
 3 12
Alpha      ACTGCTAGCT AG
Beta       ACT-CTAGCT AG
Gamma      ACTGCTAGDT AG
 3 9
Delta      GTCAGC-AG
Epislon    GACAGCTAG
Zeta       GTCAGCTAG
 3 13
Eta        ACTAGTACAG CTG
Theta      ACTAGTACAG CT-
Iota       -CTACTACAG GTG
\end{verbatim}

Its more common to want to load an existing alignment, and save that, perhaps after some simple manipulation like removing certain rows or columns.

Suppose you wanted to know how many alignments the \verb|Bio.AlignIO.write()| function wrote to the handle? If your alignments were in a list like the example above, you could just use \verb|len(my_alignments)|, however you can't do that when your records come from a generator/iterator.  Therefore the \verb|Bio.AlignIO.write()| function returns the number of alignments written to the file.   

\emph{Note} - If you tell the \verb|Bio.AlignIO.write()| function to write to a file that already exists, the old file will be overwritten without any warning.


\subsection{Converting between sequence alignment file formats}
\label{sec:converting-alignments}

Converting between sequence alignment file formats with \verb|Bio.AlignIO| works
in the same way as converting between sequence file formats with \verb|Bio.SeqIO|
(Section~\ref{sec:SeqIO-conversion}). We load generally the alignment(s) using
\verb|Bio.AlignIO.parse()| and then save them using the \verb|Bio.AlignIO.write()|
-- or just use the \verb|Bio.AlignIO.convert()| helper function.

For this example, we'll load the PFAM/Stockholm format file used earlier and save it as a Clustal W format file:

\begin{verbatim}
from Bio import AlignIO
count = AlignIO.convert("PF05371_seed.sth", "stockholm", "PF05371_seed.aln", "clustal")
print("Converted %i alignments" % count)
\end{verbatim}

Or, using \verb|Bio.AlignIO.parse()| and \verb|Bio.AlignIO.write()|:

\begin{verbatim}
from Bio import AlignIO
alignments = AlignIO.parse("PF05371_seed.sth", "stockholm")
count = AlignIO.write(alignments, "PF05371_seed.aln", "clustal")
print("Converted %i alignments" % count)
\end{verbatim}

The \verb|Bio.AlignIO.write()| function expects to be given multiple alignment objects.  In the example above we gave it the alignment iterator returned by \verb|Bio.AlignIO.parse()|.

In this case, we know there is only one alignment in the file so we could have used \verb|Bio.AlignIO.read()| instead, but notice we have to pass this alignment to \verb|Bio.AlignIO.write()| as a single element list:

\begin{verbatim}
from Bio import AlignIO
alignment = AlignIO.read("PF05371_seed.sth", "stockholm")
AlignIO.write([alignment], "PF05371_seed.aln", "clustal")
\end{verbatim}

Either way, you should end up with the same new Clustal W format file ``PF05371\_seed.aln'' with the following content:

\begin{verbatim}
CLUSTAL X (1.81) multiple sequence alignment


COATB_BPIKE/30-81                   AEPNAATNYATEAMDSLKTQAIDLISQTWPVVTTVVVAGLVIRLFKKFSS
Q9T0Q8_BPIKE/1-52                   AEPNAATNYATEAMDSLKTQAIDLISQTWPVVTTVVVAGLVIKLFKKFVS
COATB_BPI22/32-83                   DGTSTATSYATEAMNSLKTQATDLIDQTWPVVTSVAVAGLAIRLFKKFSS
COATB_BPM13/24-72                   AEGDDP---AKAAFNSLQASATEYIGYAWAMVVVIVGATIGIKLFKKFTS
COATB_BPZJ2/1-49                    AEGDDP---AKAAFDSLQASATEYIGYAWAMVVVIVGATIGIKLFKKFAS
Q9T0Q9_BPFD/1-49                    AEGDDP---AKAAFDSLQASATEYIGYAWAMVVVIVGATIGIKLFKKFTS
COATB_BPIF1/22-73                   FAADDATSQAKAAFDSLTAQATEMSGYAWALVVLVVGATVGIKLFKKFVS

COATB_BPIKE/30-81                   KA
Q9T0Q8_BPIKE/1-52                   RA
COATB_BPI22/32-83                   KA
COATB_BPM13/24-72                   KA
COATB_BPZJ2/1-49                    KA
Q9T0Q9_BPFD/1-49                    KA
COATB_BPIF1/22-73                   RA
\end{verbatim}

Alternatively, you could make a PHYLIP format file which we'll name ``PF05371\_seed.phy'':

\begin{verbatim}
from Bio import AlignIO
AlignIO.convert("PF05371_seed.sth", "stockholm", "PF05371_seed.phy", "phylip")
\end{verbatim}

This time the output looks like this:

\begin{verbatim}
 7 52
COATB_BPIK AEPNAATNYA TEAMDSLKTQ AIDLISQTWP VVTTVVVAGL VIRLFKKFSS
Q9T0Q8_BPI AEPNAATNYA TEAMDSLKTQ AIDLISQTWP VVTTVVVAGL VIKLFKKFVS
COATB_BPI2 DGTSTATSYA TEAMNSLKTQ ATDLIDQTWP VVTSVAVAGL AIRLFKKFSS
COATB_BPM1 AEGDDP---A KAAFNSLQAS ATEYIGYAWA MVVVIVGATI GIKLFKKFTS
COATB_BPZJ AEGDDP---A KAAFDSLQAS ATEYIGYAWA MVVVIVGATI GIKLFKKFAS
Q9T0Q9_BPF AEGDDP---A KAAFDSLQAS ATEYIGYAWA MVVVIVGATI GIKLFKKFTS
COATB_BPIF FAADDATSQA KAAFDSLTAQ ATEMSGYAWA LVVLVVGATV GIKLFKKFVS

           KA
           RA
           KA
           KA
           KA
           KA
           RA
\end{verbatim}

One of the big handicaps of the original PHYLIP alignment file format is
that the sequence identifiers are strictly truncated at ten characters.
In this example, as you can see the resulting names are still unique -
but they are not very readable. As a result, a more relaxed variant of
the original PHYLIP format is now quite widely used:

\begin{verbatim}
from Bio import AlignIO
AlignIO.convert("PF05371_seed.sth", "stockholm", "PF05371_seed.phy", "phylip-relaxed")
\end{verbatim}

This time the output looks like this, using a longer indentation to
allow all the identifers to be given in full::

\begin{verbatim}
 7 52
COATB_BPIKE/30-81  AEPNAATNYA TEAMDSLKTQ AIDLISQTWP VVTTVVVAGL VIRLFKKFSS
Q9T0Q8_BPIKE/1-52  AEPNAATNYA TEAMDSLKTQ AIDLISQTWP VVTTVVVAGL VIKLFKKFVS
COATB_BPI22/32-83  DGTSTATSYA TEAMNSLKTQ ATDLIDQTWP VVTSVAVAGL AIRLFKKFSS
COATB_BPM13/24-72  AEGDDP---A KAAFNSLQAS ATEYIGYAWA MVVVIVGATI GIKLFKKFTS
COATB_BPZJ2/1-49   AEGDDP---A KAAFDSLQAS ATEYIGYAWA MVVVIVGATI GIKLFKKFAS
Q9T0Q9_BPFD/1-49   AEGDDP---A KAAFDSLQAS ATEYIGYAWA MVVVIVGATI GIKLFKKFTS
COATB_BPIF1/22-73  FAADDATSQA KAAFDSLTAQ ATEMSGYAWA LVVLVVGATV GIKLFKKFVS

                   KA
                   RA
                   KA
                   KA
                   KA
                   KA
                   RA
\end{verbatim}

If you have to work with the original strict PHYLIP format, then you may need to
compress the identifers somehow -- or assign your own names or numbering system.
This following bit of code manipulates the record identifiers before saving the output:

\begin{verbatim}
from Bio import AlignIO
alignment = AlignIO.read("PF05371_seed.sth", "stockholm")
name_mapping = {}
for i, record in enumerate(alignment):
    name_mapping[i] = record.id
    record.id = "seq%i" % i
print(name_mapping)

AlignIO.write([alignment], "PF05371_seed.phy", "phylip")
\end{verbatim}

\noindent This code used a Python dictionary to record a simple mapping from the new sequence system to the original identifier:
\begin{verbatim}
{0: 'COATB_BPIKE/30-81', 1: 'Q9T0Q8_BPIKE/1-52', 2: 'COATB_BPI22/32-83', ...}
\end{verbatim}

\noindent Here is the new (strict) PHYLIP format output:
\begin{verbatim}
 7 52
seq0       AEPNAATNYA TEAMDSLKTQ AIDLISQTWP VVTTVVVAGL VIRLFKKFSS
seq1       AEPNAATNYA TEAMDSLKTQ AIDLISQTWP VVTTVVVAGL VIKLFKKFVS
seq2       DGTSTATSYA TEAMNSLKTQ ATDLIDQTWP VVTSVAVAGL AIRLFKKFSS
seq3       AEGDDP---A KAAFNSLQAS ATEYIGYAWA MVVVIVGATI GIKLFKKFTS
seq4       AEGDDP---A KAAFDSLQAS ATEYIGYAWA MVVVIVGATI GIKLFKKFAS
seq5       AEGDDP---A KAAFDSLQAS ATEYIGYAWA MVVVIVGATI GIKLFKKFTS
seq6       FAADDATSQA KAAFDSLTAQ ATEMSGYAWA LVVLVVGATV GIKLFKKFVS

           KA
           RA
           KA
           KA
           KA
           KA
           RA
\end{verbatim}

\noindent In general, because of the identifier limitation, working with
\textit{strict} PHYLIP file formats shouldn't be your first choice. 
Using the PFAM/Stockholm format on the other hand allows you to record a lot of additional annotation too.

\subsection{Getting your alignment objects as formatted strings}
\label{sec:alignment-format-method}
The \verb|Bio.AlignIO| interface is based on handles, which means if you want to get your alignment(s) into a string in a particular file format you need to do a little bit more work (see below).  
However, you will probably prefer to take advantage of the alignment object's \verb|format()| method.
This takes a single mandatory argument, a lower case string which is supported by \verb|Bio.AlignIO| as an output format.  For example:

\begin{verbatim}
from Bio import AlignIO
alignment = AlignIO.read("PF05371_seed.sth", "stockholm")
print(alignment.format("clustal"))
\end{verbatim}

As described in Section~\ref{sec:SeqRecord-format}, the \verb|SeqRecord| object has a similar method using output formats supported by \verb|Bio.SeqIO|.

Internally the \verb|format()| method is using the \verb|StringIO| string based handle and calling
\verb|Bio.AlignIO.write()|.  You can do this in your own code if for example you are using an
older version of Biopython:

\begin{verbatim}
from Bio import AlignIO
from StringIO import StringIO

alignments = AlignIO.parse("PF05371_seed.sth", "stockholm")

out_handle = StringIO()
AlignIO.write(alignments, out_handle, "clustal")
clustal_data = out_handle.getvalue()

print(clustal_data)
\end{verbatim}

\section{Manipulating Alignments}
\label{sec:manipulating-alignments}

Now that we've covered loading and saving alignments, we'll look at what else you can do
with them.

\subsection{Slicing alignments}
First of all, in some senses the alignment objects act like a Python \verb|list| of
\verb|SeqRecord| objects (the rows). With this model in mind hopefully the actions
of \verb|len()| (the number of rows) and iteration (each row as a \verb|SeqRecord|)
make sense:

%doctest examples
\begin{verbatim}
>>> from Bio import AlignIO
>>> alignment = AlignIO.read("PF05371_seed.sth", "stockholm")
>>> print("Number of rows: %i" % len(alignment))
Number of rows: 7
>>> for record in alignment:
...     print("%s - %s" % (record.seq, record.id))
AEPNAATNYATEAMDSLKTQAIDLISQTWPVVTTVVVAGLVIRLFKKFSSKA - COATB_BPIKE/30-81
AEPNAATNYATEAMDSLKTQAIDLISQTWPVVTTVVVAGLVIKLFKKFVSRA - Q9T0Q8_BPIKE/1-52
DGTSTATSYATEAMNSLKTQATDLIDQTWPVVTSVAVAGLAIRLFKKFSSKA - COATB_BPI22/32-83
AEGDDP---AKAAFNSLQASATEYIGYAWAMVVVIVGATIGIKLFKKFTSKA - COATB_BPM13/24-72
AEGDDP---AKAAFDSLQASATEYIGYAWAMVVVIVGATIGIKLFKKFASKA - COATB_BPZJ2/1-49
AEGDDP---AKAAFDSLQASATEYIGYAWAMVVVIVGATIGIKLFKKFTSKA - Q9T0Q9_BPFD/1-49
FAADDATSQAKAAFDSLTAQATEMSGYAWALVVLVVGATVGIKLFKKFVSRA - COATB_BPIF1/22-73
\end{verbatim}

You can also use the list-like \verb|append| and \verb|extend| methods to add
more rows to the alignment (as \verb|SeqRecord| objects). Keeping the list
metaphor in mind, simple slicing of the alignment should also make sense -
it selects some of the rows giving back another alignment object:

%cont-doctest
\begin{verbatim}
>>> print(alignment)
SingleLetterAlphabet() alignment with 7 rows and 52 columns
AEPNAATNYATEAMDSLKTQAIDLISQTWPVVTTVVVAGLVIRL...SKA COATB_BPIKE/30-81
AEPNAATNYATEAMDSLKTQAIDLISQTWPVVTTVVVAGLVIKL...SRA Q9T0Q8_BPIKE/1-52
DGTSTATSYATEAMNSLKTQATDLIDQTWPVVTSVAVAGLAIRL...SKA COATB_BPI22/32-83
AEGDDP---AKAAFNSLQASATEYIGYAWAMVVVIVGATIGIKL...SKA COATB_BPM13/24-72
AEGDDP---AKAAFDSLQASATEYIGYAWAMVVVIVGATIGIKL...SKA COATB_BPZJ2/1-49
AEGDDP---AKAAFDSLQASATEYIGYAWAMVVVIVGATIGIKL...SKA Q9T0Q9_BPFD/1-49
FAADDATSQAKAAFDSLTAQATEMSGYAWALVVLVVGATVGIKL...SRA COATB_BPIF1/22-73
>>> print(alignment[3:7])
SingleLetterAlphabet() alignment with 4 rows and 52 columns
AEGDDP---AKAAFNSLQASATEYIGYAWAMVVVIVGATIGIKL...SKA COATB_BPM13/24-72
AEGDDP---AKAAFDSLQASATEYIGYAWAMVVVIVGATIGIKL...SKA COATB_BPZJ2/1-49
AEGDDP---AKAAFDSLQASATEYIGYAWAMVVVIVGATIGIKL...SKA Q9T0Q9_BPFD/1-49
FAADDATSQAKAAFDSLTAQATEMSGYAWALVVLVVGATVGIKL...SRA COATB_BPIF1/22-73
\end{verbatim}

What if you wanted to select by column? Those of you who have used the NumPy
matrix or array objects won't be surprised at this - you use a double index.

%cont-doctest
\begin{verbatim}
>>> print(alignment[2, 6])
T
\end{verbatim}

\noindent Using two integer indices pulls out a single letter, short hand for this:

%cont-doctest
\begin{verbatim}
>>> print(alignment[2].seq[6])
T
\end{verbatim}

You can pull out a single column as a string like this:

%cont-doctest
\begin{verbatim}
>>> print(alignment[:, 6])
TTT---T
\end{verbatim}

You can also select a range of columns. For example, to pick out those same
three rows we extracted earlier, but take just their first six columns:

%cont-doctest
\begin{verbatim}
>>> print(alignment[3:6, :6])
SingleLetterAlphabet() alignment with 3 rows and 6 columns
AEGDDP COATB_BPM13/24-72
AEGDDP COATB_BPZJ2/1-49
AEGDDP Q9T0Q9_BPFD/1-49
\end{verbatim}

Leaving the first index as \verb|:| means take all the rows:

%cont-doctest
\begin{verbatim}
>>> print(alignment[:, :6])
SingleLetterAlphabet() alignment with 7 rows and 6 columns
AEPNAA COATB_BPIKE/30-81
AEPNAA Q9T0Q8_BPIKE/1-52
DGTSTA COATB_BPI22/32-83
AEGDDP COATB_BPM13/24-72
AEGDDP COATB_BPZJ2/1-49
AEGDDP Q9T0Q9_BPFD/1-49
FAADDA COATB_BPIF1/22-73
\end{verbatim}

This brings us to a neat way to remove a section. Notice columns
7, 8 and 9 which are gaps in three of the seven sequences:

%cont-doctest
\begin{verbatim}
>>> print(alignment[:, 6:9])
SingleLetterAlphabet() alignment with 7 rows and 3 columns
TNY COATB_BPIKE/30-81
TNY Q9T0Q8_BPIKE/1-52
TSY COATB_BPI22/32-83
--- COATB_BPM13/24-72
--- COATB_BPZJ2/1-49
--- Q9T0Q9_BPFD/1-49
TSQ COATB_BPIF1/22-73
\end{verbatim}

\noindent Again, you can slice to get everything after the ninth column:

%cont-doctest
\begin{verbatim}
>>> print(alignment[:, 9:])
SingleLetterAlphabet() alignment with 7 rows and 43 columns
ATEAMDSLKTQAIDLISQTWPVVTTVVVAGLVIRLFKKFSSKA COATB_BPIKE/30-81
ATEAMDSLKTQAIDLISQTWPVVTTVVVAGLVIKLFKKFVSRA Q9T0Q8_BPIKE/1-52
ATEAMNSLKTQATDLIDQTWPVVTSVAVAGLAIRLFKKFSSKA COATB_BPI22/32-83
AKAAFNSLQASATEYIGYAWAMVVVIVGATIGIKLFKKFTSKA COATB_BPM13/24-72
AKAAFDSLQASATEYIGYAWAMVVVIVGATIGIKLFKKFASKA COATB_BPZJ2/1-49
AKAAFDSLQASATEYIGYAWAMVVVIVGATIGIKLFKKFTSKA Q9T0Q9_BPFD/1-49
AKAAFDSLTAQATEMSGYAWALVVLVVGATVGIKLFKKFVSRA COATB_BPIF1/22-73
\end{verbatim}

\noindent Now, the interesting thing is that addition of alignment objects works
by column. This lets you do this as a way to remove a block of columns:

%cont-doctest
\begin{verbatim}
>>> edited = alignment[:, :6] + alignment[:, 9:]
>>> print(edited)
SingleLetterAlphabet() alignment with 7 rows and 49 columns
AEPNAAATEAMDSLKTQAIDLISQTWPVVTTVVVAGLVIRLFKKFSSKA COATB_BPIKE/30-81
AEPNAAATEAMDSLKTQAIDLISQTWPVVTTVVVAGLVIKLFKKFVSRA Q9T0Q8_BPIKE/1-52
DGTSTAATEAMNSLKTQATDLIDQTWPVVTSVAVAGLAIRLFKKFSSKA COATB_BPI22/32-83
AEGDDPAKAAFNSLQASATEYIGYAWAMVVVIVGATIGIKLFKKFTSKA COATB_BPM13/24-72
AEGDDPAKAAFDSLQASATEYIGYAWAMVVVIVGATIGIKLFKKFASKA COATB_BPZJ2/1-49
AEGDDPAKAAFDSLQASATEYIGYAWAMVVVIVGATIGIKLFKKFTSKA Q9T0Q9_BPFD/1-49
FAADDAAKAAFDSLTAQATEMSGYAWALVVLVVGATVGIKLFKKFVSRA COATB_BPIF1/22-73
\end{verbatim}

Another common use of alignment addition would be to combine alignments for
several different genes into a meta-alignment. Watch out though - the identifiers
need to match up (see Section~\ref{sec:SeqRecord-addition} for how adding
\verb|SeqRecord| objects works). You may find it helpful to first sort the
alignment rows alphabetically by id:

%cont-doctest
\begin{verbatim}
>>> edited.sort()
>>> print(edited)
SingleLetterAlphabet() alignment with 7 rows and 49 columns
DGTSTAATEAMNSLKTQATDLIDQTWPVVTSVAVAGLAIRLFKKFSSKA COATB_BPI22/32-83
FAADDAAKAAFDSLTAQATEMSGYAWALVVLVVGATVGIKLFKKFVSRA COATB_BPIF1/22-73
AEPNAAATEAMDSLKTQAIDLISQTWPVVTTVVVAGLVIRLFKKFSSKA COATB_BPIKE/30-81
AEGDDPAKAAFNSLQASATEYIGYAWAMVVVIVGATIGIKLFKKFTSKA COATB_BPM13/24-72
AEGDDPAKAAFDSLQASATEYIGYAWAMVVVIVGATIGIKLFKKFASKA COATB_BPZJ2/1-49
AEPNAAATEAMDSLKTQAIDLISQTWPVVTTVVVAGLVIKLFKKFVSRA Q9T0Q8_BPIKE/1-52
AEGDDPAKAAFDSLQASATEYIGYAWAMVVVIVGATIGIKLFKKFTSKA Q9T0Q9_BPFD/1-49
\end{verbatim}

\noindent Note that you can only add two alignments together if they
have the same number of rows.

\subsection{Alignments as arrays}
Depending on what you are doing, it can be more useful to turn the alignment
object into an array of letters -- and you can do this with NumPy:

%This example fails under PyPy 2.0, https://bugs.pypy.org/issue1546
%doctest examples lib:numpy
\begin{verbatim}
>>> import numpy as np
>>> from Bio import AlignIO
>>> alignment = AlignIO.read("PF05371_seed.sth", "stockholm")
>>> align_array = np.array([list(rec) for rec in alignment], np.character)
>>> align_array.shape
(7, 52)
\end{verbatim}

If you will be working heavily with the columns, you can tell NumPy to store
the array by column (as in Fortran) rather then its default of by row (as in C):

\begin{verbatim}
>>> align_array = np.array([list(rec) for rec in alignment], np.character, order="F")
\end{verbatim}

Note that this leaves the original Biopython alignment object and the NumPy array
in memory as separate objects - editing one will not update the other!

\section{Alignment Tools}
\label{sec:alignment-tools}

There are \emph{lots} of algorithms out there for aligning sequences, both pairwise alignments
and multiple sequence alignments. These calculations are relatively slow, and you generally
wouldn't want to write such an algorithm in Python. Instead, you can use Biopython to invoke
a command line tool on your behalf. Normally you would:
\begin{enumerate}
\item Prepare an input file of your unaligned sequences, typically this will be a FASTA file
      which you might create using \verb|Bio.SeqIO| (see Chapter~\ref{chapter:Bio.SeqIO}).
\item Call the command line tool to process this input file, typically via one of Biopython's
      command line wrappers (which we'll discuss here).
\item Read the output from the tool, i.e. your aligned sequences, typically using
      \verb|Bio.AlignIO| (see earlier in this chapter). 
\end{enumerate}

All the command line wrappers we're going to talk about in this chapter follow the same style.
You create a command line object specifying the options (e.g. the input filename and the
output filename), then invoke this command line via a Python operating system call (e.g.
using the \texttt{subprocess} module).

Most of these wrappers are defined in the \verb|Bio.Align.Applications| module:

\begin{verbatim}
>>> import Bio.Align.Applications
>>> dir(Bio.Align.Applications)
...
['ClustalwCommandline', 'DialignCommandline', 'MafftCommandline', 'MuscleCommandline',
'PrankCommandline', 'ProbconsCommandline', 'TCoffeeCommandline' ...]
\end{verbatim}

\noindent (Ignore the entries starting with an underscore -- these have
special meaning in Python.)
The module \verb|Bio.Emboss.Applications| has wrappers for some of the
\href{http://emboss.sourceforge.net/}{EMBOSS suite}, including
\texttt{needle} and \texttt{water}, which are described below in
Section~\ref{seq:emboss-needle-water}, and wrappers for the EMBOSS
packaged versions of the PHYLIP tools (which EMBOSS refer to as one
of their EMBASSY packages - third party tools with an EMBOSS style
interface).
We won't explore all these alignment tools here in the section, just a
sample, but the same principles apply.

\subsection{ClustalW}
\label{sec:align_clustal}
ClustalW is a popular command line tool for multiple sequence alignment
(there is also a graphical interface called ClustalX). Biopython's
\verb|Bio.Align.Applications| module has a wrapper for this alignment tool
(and several others).

Before trying to use ClustalW from within Python, you should first try running
the ClustalW tool yourself by hand at the command line, to familiarise
yourself the other options. You'll find the Biopython wrapper is very
faithful to the actual command line API:

\begin{verbatim}
>>> from Bio.Align.Applications import ClustalwCommandline
>>> help(ClustalwCommandline)
...
\end{verbatim}

For the most basic usage, all you need is to have a FASTA input file, such as
\href{http://biopython.org/DIST/docs/tutorial/examples/opuntia.fasta}{opuntia.fasta}
(available online or in the Doc/examples subdirectory of the Biopython source
code). This is a small FASTA file containing seven prickly-pear DNA sequences
(from the cactus family \textit{Opuntia}).

By default ClustalW will generate an alignment and guide tree file with names
based on the input FASTA file, in this case \texttt{opuntia.aln} and
\texttt{opuntia.dnd}, but you can override this or make it explicit:

%doctest
\begin{verbatim}
>>> from Bio.Align.Applications import ClustalwCommandline
>>> cline = ClustalwCommandline("clustalw2", infile="opuntia.fasta")
>>> print(cline)
clustalw2 -infile=opuntia.fasta
\end{verbatim}

Notice here we have given the executable name as \texttt{clustalw2},
indicating we have version two installed, which has a different filename to
version one (\texttt{clustalw}, the default). Fortunately both versions
support the same set of arguments at the command line (and indeed, should be
functionally identical).

You may find that even though you have ClustalW installed, the above command
doesn't work -- you may get a message about ``command not found'' (especially
on Windows). This indicated that the ClustalW executable is not on your PATH
(an environment variable, a list of directories to be searched). You can
either update your PATH setting to include the location of your copy of
ClustalW tools (how you do this will depend on your OS), or simply type in
the full path of the tool. For example:

%doctest
\begin{verbatim}
>>> import os
>>> from Bio.Align.Applications import ClustalwCommandline
>>> clustalw_exe = r"C:\Program Files\new clustal\clustalw2.exe"
>>> clustalw_cline = ClustalwCommandline(clustalw_exe, infile="opuntia.fasta")
\end{verbatim}
%Don't run it in the doctest
\begin{verbatim}
>>> assert os.path.isfile(clustalw_exe), "Clustal W executable missing"
>>> stdout, stderr = clustalw_cline()
\end{verbatim}

\noindent Remember, in Python strings \verb|\n| and \verb|\t| are by default
interpreted as a new line and a tab -- which is why we're put a letter
``r'' at the start for a raw string that isn't translated in this way.
This is generally good practice when specifying a Windows style file name.

Internally this uses the
\verb|subprocess| module which is now the recommended way to run another
program in Python. This replaces older options like the \verb|os.system()|
and the \verb|os.popen*| functions.

Now, at this point it helps to know about how command line tools ``work''.
When you run a tool at the command line, it will often print text output
directly to screen. This text can be captured or redirected, via
two ``pipes'', called standard output (the normal results) and standard
error (for error messages and debug messages). There is also standard
input, which is any text fed into the tool. These names get shortened
to stdin, stdout and stderr. When the tool finishes, it has a return
code (an integer), which by convention is zero for success.

When you run the command line tool like this via the Biopython wrapper,
it will wait for it to finish, and check the return code. If this is
non zero (indicating an error), an exception is raised. The wrapper
then returns two strings, stdout and stderr.

In the case of ClustalW, when run at the command line all the important
output is written directly to the output files. Everything normally printed to
screen while you wait (via stdout or stderr) is boring and can be
ignored (assuming it worked).

What we care about are the two output files, the alignment and the guide
tree. We didn't tell ClustalW what filenames to use, but it defaults to
picking names based on the input file. In this case the output should be
in the file \verb|opuntia.aln|.
You should be able to work out how to read in the alignment using
\verb|Bio.AlignIO| by now:

\begin{verbatim}
>>> from Bio import AlignIO
>>> align = AlignIO.read("opuntia.aln", "clustal")
>>> print(align)
SingleLetterAlphabet() alignment with 7 rows and 906 columns
TATACATTAAAGAAGGGGGATGCGGATAAATGGAAAGGCGAAAG...AGA gi|6273285|gb|AF191659.1|AF191
TATACATTAAAGAAGGGGGATGCGGATAAATGGAAAGGCGAAAG...AGA gi|6273284|gb|AF191658.1|AF191
TATACATTAAAGAAGGGGGATGCGGATAAATGGAAAGGCGAAAG...AGA gi|6273287|gb|AF191661.1|AF191
TATACATAAAAGAAGGGGGATGCGGATAAATGGAAAGGCGAAAG...AGA gi|6273286|gb|AF191660.1|AF191
TATACATTAAAGGAGGGGGATGCGGATAAATGGAAAGGCGAAAG...AGA gi|6273290|gb|AF191664.1|AF191
TATACATTAAAGGAGGGGGATGCGGATAAATGGAAAGGCGAAAG...AGA gi|6273289|gb|AF191663.1|AF191
TATACATTAAAGGAGGGGGATGCGGATAAATGGAAAGGCGAAAG...AGA gi|6273291|gb|AF191665.1|AF191
\end{verbatim}

In case you are interested (and this is an aside from the main thrust of this
chapter), the \texttt{opuntia.dnd} file ClustalW creates is just a standard
Newick tree file, and \verb|Bio.Phylo| can parse these:

\begin{verbatim}
>>> from Bio import Phylo
>>> tree = Phylo.read("opuntia.dnd", "newick")
>>> Phylo.draw_ascii(tree)
                             _______________ gi|6273291|gb|AF191665.1|AF191665
  __________________________|
 |                          |   ______ gi|6273290|gb|AF191664.1|AF191664
 |                          |__|
 |                             |_____ gi|6273289|gb|AF191663.1|AF191663
 |
_|_________________ gi|6273287|gb|AF191661.1|AF191661
 |
 |__________ gi|6273286|gb|AF191660.1|AF191660
 |
 |    __ gi|6273285|gb|AF191659.1|AF191659
 |___|
     | gi|6273284|gb|AF191658.1|AF191658

\end{verbatim}

\noindent Chapter \ref{sec:Phylo} covers Biopython's support for phylogenetic trees in more
depth.

\subsection{MUSCLE}
MUSCLE is a more recent multiple sequence alignment tool than ClustalW, and
Biopython also has a wrapper for it under the \verb|Bio.Align.Applications|
module. As before, we recommend you try using MUSCLE from the command line before
trying it from within Python, as the Biopython wrapper is very faithful to the
actual command line API:

\begin{verbatim}
>>> from Bio.Align.Applications import MuscleCommandline
>>> help(MuscleCommandline)
...
\end{verbatim}

For the most basic usage, all you need is to have a FASTA input file, such as
\href{http://biopython.org/DIST/docs/tutorial/examples/opuntia.fasta}{opuntia.fasta}
(available online or in the Doc/examples subdirectory of the Biopython source
code). You can then tell MUSCLE to read in this FASTA file, and write the
alignment to an output file:

%doctest
\begin{verbatim}
>>> from Bio.Align.Applications import MuscleCommandline
>>> cline = MuscleCommandline(input="opuntia.fasta", out="opuntia.txt")
>>> print(cline)
muscle -in opuntia.fasta -out opuntia.txt
\end{verbatim}

Note that MUSCLE uses ``-in'' and ``-out'' but in Biopython we have to use
``input'' and ``out'' as the keyword arguments or property names. This is
because ``in'' is a reserved word in Python.

By default MUSCLE will output the alignment as a FASTA file (using gapped
sequences). The \verb|Bio.AlignIO| module should be able to read this
alignment using \texttt{format="fasta"}.
You can also ask for ClustalW-like output:

%doctest
\begin{verbatim}
>>> from Bio.Align.Applications import MuscleCommandline
>>> cline = MuscleCommandline(input="opuntia.fasta", out="opuntia.aln", clw=True)
>>> print(cline)
muscle -in opuntia.fasta -out opuntia.aln -clw
\end{verbatim}

Or, strict ClustalW output where the original ClustalW header line is
used for maximum compatibility:

%doctest
\begin{verbatim}
>>> from Bio.Align.Applications import MuscleCommandline
>>> cline = MuscleCommandline(input="opuntia.fasta", out="opuntia.aln", clwstrict=True)
>>> print(cline)
muscle -in opuntia.fasta -out opuntia.aln -clwstrict
\end{verbatim}

\noindent The \verb|Bio.AlignIO| module should be able to read these alignments
using \texttt{format="clustal"}.

MUSCLE can also output in GCG MSF format (using the \texttt{msf} argument), but
Biopython can't currently parse that, or using HTML which would give a human
readable web page (not suitable for parsing).

You can also set the other optional parameters, for example the maximum number
of iterations. See the built in help for details.

You would then run MUSCLE command line string as described above for
ClustalW, and parse the output using \verb|Bio.AlignIO| to get an
alignment object.

\subsection{MUSCLE using stdout}

Using a MUSCLE command line as in the examples above will write the alignment
to a file. This means there will be no important information written to the
standard out (stdout) or standard error (stderr) handles. However, by default
MUSCLE will write the alignment to standard output (stdout). We can take
advantage of this to avoid having a temporary output file! For example:

%doctest
\begin{verbatim}
>>> from Bio.Align.Applications import MuscleCommandline
>>> muscle_cline = MuscleCommandline(input="opuntia.fasta")
>>> print(muscle_cline)
muscle -in opuntia.fasta
\end{verbatim}

If we run this via the wrapper, we get back the output as a string. In order
to parse this we can use \verb|StringIO| to turn it into a handle.
Remember that MUSCLE defaults to using FASTA as the output format:

\begin{verbatim}
>>> from Bio.Align.Applications import MuscleCommandline
>>> muscle_cline = MuscleCommandline(input="opuntia.fasta")
>>> stdout, stderr = muscle_cline()
>>> from StringIO import StringIO
>>> from Bio import AlignIO
>>> align = AlignIO.read(StringIO(stdout), "fasta")
>>> print(align)
SingleLetterAlphabet() alignment with 7 rows and 906 columns
TATACATTAAAGGAGGGGGATGCGGATAAATGGAAAGGCGAAAG...AGA gi|6273289|gb|AF191663.1|AF191663
TATACATTAAAGGAGGGGGATGCGGATAAATGGAAAGGCGAAAG...AGA gi|6273291|gb|AF191665.1|AF191665
TATACATTAAAGGAGGGGGATGCGGATAAATGGAAAGGCGAAAG...AGA gi|6273290|gb|AF191664.1|AF191664
TATACATTAAAGAAGGGGGATGCGGATAAATGGAAAGGCGAAAG...AGA gi|6273287|gb|AF191661.1|AF191661
TATACATAAAAGAAGGGGGATGCGGATAAATGGAAAGGCGAAAG...AGA gi|6273286|gb|AF191660.1|AF191660
TATACATTAAAGAAGGGGGATGCGGATAAATGGAAAGGCGAAAG...AGA gi|6273285|gb|AF191659.1|AF191659
TATACATTAAAGAAGGGGGATGCGGATAAATGGAAAGGCGAAAG...AGA gi|6273284|gb|AF191658.1|AF191658
\end{verbatim}

The above approach is fairly simple, but if you are dealing with very large output
text the fact that all of stdout and stderr is loaded into memory as a string can
be a potential drawback. Using the \verb|subprocess| module we can work directly
with handles instead:

\begin{verbatim}
>>> import subprocess
>>> from Bio.Align.Applications import MuscleCommandline
>>> muscle_cline = MuscleCommandline(input="opuntia.fasta")
>>> child = subprocess.Popen(str(muscle_cline),
...                          stdout=subprocess.PIPE,
...                          stderr=subprocess.PIPE,
...                          shell=(sys.platform!="win32"))
>>> from Bio import AlignIO
>>> align = AlignIO.read(child.stdout, "fasta")
>>> print(align)
SingleLetterAlphabet() alignment with 7 rows and 906 columns
TATACATTAAAGGAGGGGGATGCGGATAAATGGAAAGGCGAAAG...AGA gi|6273289|gb|AF191663.1|AF191663
TATACATTAAAGGAGGGGGATGCGGATAAATGGAAAGGCGAAAG...AGA gi|6273291|gb|AF191665.1|AF191665
TATACATTAAAGGAGGGGGATGCGGATAAATGGAAAGGCGAAAG...AGA gi|6273290|gb|AF191664.1|AF191664
TATACATTAAAGAAGGGGGATGCGGATAAATGGAAAGGCGAAAG...AGA gi|6273287|gb|AF191661.1|AF191661
TATACATAAAAGAAGGGGGATGCGGATAAATGGAAAGGCGAAAG...AGA gi|6273286|gb|AF191660.1|AF191660
TATACATTAAAGAAGGGGGATGCGGATAAATGGAAAGGCGAAAG...AGA gi|6273285|gb|AF191659.1|AF191659
TATACATTAAAGAAGGGGGATGCGGATAAATGGAAAGGCGAAAG...AGA gi|6273284|gb|AF191658.1|AF191658
\end{verbatim}

\subsection{MUSCLE using stdin and stdout}

We don't actually \emph{need} to have our FASTA input sequences prepared in a file,
because by default MUSCLE will read in the input sequence from standard input!
Note this is a bit more advanced and fiddly, so don't bother with this technique
unless you need to.

First, we'll need some unaligned sequences in memory as \verb|SeqRecord| objects.
For this demonstration I'm going to use a filtered version of the original FASTA
file (using a generator expression), taking just six of the seven sequences:

%doctest
\begin{verbatim}
>>> from Bio import SeqIO
>>> records = (r for r in SeqIO.parse("opuntia.fasta", "fasta") if len(r) < 900)
\end{verbatim}

Then we create the MUSCLE command line, leaving the input and output to their
defaults (stdin and stdout). I'm also going to ask for strict ClustalW format
as for the output.

%doctest
\begin{verbatim}
>>> from Bio.Align.Applications import MuscleCommandline
>>> muscle_cline = MuscleCommandline(clwstrict=True)
>>> print(muscle_cline)
muscle -clwstrict
\end{verbatim}

Now for the fiddly bits using the \verb|subprocess| module, stdin and stdout:

\begin{verbatim}
>>> import subprocess
>>> import sys
>>> child = subprocess.Popen(str(cline),
...                          stdin=subprocess.PIPE,
...                          stdout=subprocess.PIPE,
...                          stderr=subprocess.PIPE,
...                          universal_newlines=True,
...                          shell=(sys.platform!="win32"))                     
\end{verbatim}

That should start MUSCLE, but it will be sitting waiting for its FASTA input
sequences, which we must supply via its stdin handle:

\begin{verbatim}
>>> SeqIO.write(records, child.stdin, "fasta")
6
>>> child.stdin.close()
\end{verbatim}

After writing the six sequences to the handle, MUSCLE will still be waiting
to see if that is all the FASTA sequences or not -- so we must signal that
this is all the input data by closing the handle. At that point MUSCLE should
start to run, and we can ask for the output:

\begin{verbatim}
>>> from Bio import AlignIO
>>> align = AlignIO.read(child.stdout, "clustal")
>>> print(align)
SingleLetterAlphabet() alignment with 6 rows and 900 columns
TATACATTAAAGGAGGGGGATGCGGATAAATGGAAAGGCGAAAG...AGA gi|6273290|gb|AF191664.1|AF19166
TATACATTAAAGGAGGGGGATGCGGATAAATGGAAAGGCGAAAG...AGA gi|6273289|gb|AF191663.1|AF19166
TATACATTAAAGAAGGGGGATGCGGATAAATGGAAAGGCGAAAG...AGA gi|6273287|gb|AF191661.1|AF19166
TATACATAAAAGAAGGGGGATGCGGATAAATGGAAAGGCGAAAG...AGA gi|6273286|gb|AF191660.1|AF19166
TATACATTAAAGAAGGGGGATGCGGATAAATGGAAAGGCGAAAG...AGA gi|6273285|gb|AF191659.1|AF19165
TATACATTAAAGAAGGGGGATGCGGATAAATGGAAAGGCGAAAG...AGA gi|6273284|gb|AF191658.1|AF19165
\end{verbatim}

Wow! There we are with a new alignment of just the six records, without having created
a temporary FASTA input file, or a temporary alignment output file. However, a word of
caution: Dealing with errors with this style of calling external programs is much more
complicated.
It also becomes far harder to diagnose problems, because you can't try running MUSCLE
manually outside of Biopython (because you don't have the input file to supply).
There can also be subtle cross platform issues (e.g. Windows versus Linux,
Python 2 versus Python 3), and how
you run your script can have an impact (e.g. at the command line, from IDLE or an
IDE, or as a GUI script). These are all generic Python issues though, and not
specific to Biopython.

If you find working directly with \texttt{subprocess} like this scary, there is an
alternative. If you execute the tool with \texttt{muscle\_cline()} you can supply
any standard input as a big string, \texttt{muscle\_cline(stdin=...)}. So,
provided your data isn't very big, you can prepare the FASTA input in memory as
a string using \texttt{StringIO} (see Section~\ref{sec:appendix-handles}):

%doctest
\begin{verbatim}
>>> from Bio import SeqIO
>>> records = (r for r in SeqIO.parse("opuntia.fasta", "fasta") if len(r) < 900)
>>> from StringIO import StringIO
>>> handle = StringIO()
>>> SeqIO.write(records, handle, "fasta")
6
>>> data = handle.getvalue()
\end{verbatim}

\noindent You can then run the tool and parse the alignment as follows:

%not a doctest as can't assume the MUSCLE binary is present
\begin{verbatim}
>>> stdout, stderr = muscle_cline(stdin=data)
>>> from Bio import AlignIO
>>> align = AlignIO.read(StringIO(stdout), "clustal")
>>> print(align)
SingleLetterAlphabet() alignment with 6 rows and 900 columns
TATACATTAAAGGAGGGGGATGCGGATAAATGGAAAGGCGAAAG...AGA gi|6273290|gb|AF191664.1|AF19166
TATACATTAAAGGAGGGGGATGCGGATAAATGGAAAGGCGAAAG...AGA gi|6273289|gb|AF191663.1|AF19166
TATACATTAAAGAAGGGGGATGCGGATAAATGGAAAGGCGAAAG...AGA gi|6273287|gb|AF191661.1|AF19166
TATACATAAAAGAAGGGGGATGCGGATAAATGGAAAGGCGAAAG...AGA gi|6273286|gb|AF191660.1|AF19166
TATACATTAAAGAAGGGGGATGCGGATAAATGGAAAGGCGAAAG...AGA gi|6273285|gb|AF191659.1|AF19165
TATACATTAAAGAAGGGGGATGCGGATAAATGGAAAGGCGAAAG...AGA gi|6273284|gb|AF191658.1|AF19165
\end{verbatim}

You might find this easier, but it does require more memory (RAM) for the strings
used for the input FASTA and output Clustal formatted data.

\subsection{EMBOSS needle and water}
\label{seq:emboss-needle-water}
The \href{http://emboss.sourceforge.net/}{EMBOSS} suite includes the \texttt{water} and
\texttt{needle} tools for Smith-Waterman algorithm local alignment, and Needleman-Wunsch
global alignment. The tools share the same style interface, so switching between the two
is trivial -- we'll just use \texttt{needle} here.

Suppose you want to do a global pairwise alignment between two sequences, prepared in
FASTA format as follows:

\begin{verbatim}
>HBA_HUMAN
MVLSPADKTNVKAAWGKVGAHAGEYGAEALERMFLSFPTTKTYFPHFDLSHGSAQVKGHG
KKVADALTNAVAHVDDMPNALSALSDLHAHKLRVDPVNFKLLSHCLLVTLAAHLPAEFTP
AVHASLDKFLASVSTVLTSKYR
\end{verbatim}

\noindent in a file \texttt{alpha.fasta}, and secondly in a file \texttt{beta.fasta}:

\begin{verbatim}
>HBB_HUMAN
MVHLTPEEKSAVTALWGKVNVDEVGGEALGRLLVVYPWTQRFFESFGDLSTPDAVMGNPK
VKAHGKKVLGAFSDGLAHLDNLKGTFATLSELHCDKLHVDPENFRLLGNVLVCVLAHHFG
KEFTPPVQAAYQKVVAGVANALAHKYH
\end{verbatim}

Let's start by creating a complete \texttt{needle} command line object in one go:

%doctest
\begin{verbatim}
>>> from Bio.Emboss.Applications import NeedleCommandline
>>> needle_cline = NeedleCommandline(asequence="alpha.faa", bsequence="beta.faa",
...                                  gapopen=10, gapextend=0.5, outfile="needle.txt")
>>> print(needle_cline)
needle -outfile=needle.txt -asequence=alpha.faa -bsequence=beta.faa -gapopen=10 -gapextend=0.5
\end{verbatim}

Why not try running this by hand at the command prompt? You should see it does a
pairwise comparison and records the output in the file \texttt{needle.txt} (in the
default EMBOSS alignment file format).

Even if you have EMBOSS installed, running this command may not work -- you
might get a message about ``command not found'' (especially on Windows). This
probably means that the EMBOSS tools are not on your PATH environment
variable. You can either update your PATH setting, or simply tell Biopython
the full path to the tool, for example:

%doctest
\begin{verbatim}
>>> from Bio.Emboss.Applications import NeedleCommandline
>>> needle_cline = NeedleCommandline(r"C:\EMBOSS\needle.exe",
...                                  asequence="alpha.faa", bsequence="beta.faa",
...                                  gapopen=10, gapextend=0.5, outfile="needle.txt")
\end{verbatim}

\noindent Remember in Python that for a default string \verb|\n| or \verb|\t| means a
new line or a tab -- which is why we're put a letter ``r'' at the start for a raw string.

At this point it might help to try running the EMBOSS tools yourself by hand at the
command line, to familiarise yourself the other options and compare them to the
Biopython help text:

\begin{verbatim}
>>> from Bio.Emboss.Applications import NeedleCommandline
>>> help(NeedleCommandline)
...
\end{verbatim}

Note that you can also specify (or change or look at) the settings like this:

%doctest
\begin{verbatim}
>>> from Bio.Emboss.Applications import NeedleCommandline
>>> needle_cline = NeedleCommandline()
>>> needle_cline.asequence="alpha.faa"
>>> needle_cline.bsequence="beta.faa"
>>> needle_cline.gapopen=10
>>> needle_cline.gapextend=0.5
>>> needle_cline.outfile="needle.txt"
>>> print(needle_cline)
needle -outfile=needle.txt -asequence=alpha.faa -bsequence=beta.faa -gapopen=10 -gapextend=0.5
>>> print(needle_cline.outfile)
needle.txt
\end{verbatim}

Next we want to use Python to run this command for us. As explained above,
for full control, we recommend you use the built in Python \texttt{subprocess}
module, but for simple usage the wrapper object usually suffices:

\begin{verbatim}
>>> stdout, stderr = needle_cline()
>>> print(stdout + stderr)
Needleman-Wunsch global alignment of two sequences
\end{verbatim}

Next we can load the output file with \verb|Bio.AlignIO| as
discussed earlier in this chapter, as the \texttt{emboss} format:

\begin{verbatim}
>>> from Bio import AlignIO
>>> align = AlignIO.read("needle.txt", "emboss")
>>> print(align)
SingleLetterAlphabet() alignment with 2 rows and 149 columns
MV-LSPADKTNVKAAWGKVGAHAGEYGAEALERMFLSFPTTKTY...KYR HBA_HUMAN
MVHLTPEEKSAVTALWGKV--NVDEVGGEALGRLLVVYPWTQRF...KYH HBB_HUMAN
\end{verbatim}

In this example, we told EMBOSS to write the output to a file, but you
\emph{can} tell it to write the output to stdout instead (useful if you
don't want a temporary output file to get rid of -- use
\texttt{stdout=True} rather than the \texttt{outfile} argument), and
also to read \emph{one} of the one of the inputs from stdin (e.g.
\texttt{asequence="stdin"}, much like in the MUSCLE example in the
section above).

This has only scratched the surface of what you can do with \texttt{needle}
and \texttt{water}. One useful trick is that the second file can contain
multiple sequences (say five), and then EMBOSS will do five pairwise
alignments.

Note - Biopython includes its own pairwise alignment code in the \verb|Bio.pairwise2|
module (written in C for speed, but with a pure Python fallback available too). This
doesn't work with alignment objects, so we have not covered it within this chapter.
See the module's docstring (built in help) for details.


% chapter7 BLAST
\include{chapter/chapter7}

% chapter8 BLAST and other sequence search tools (\textit{experimental code})
\include{chapter/chapter8}

% chapter9 Accessing NCBI's Entrez databases
\include{chapter/chapter9}

% chapter10 Swiss-Prot and ExPASy
\include{chapter/chapter10}

% chapter11 Going 3D: The PDB module
\include{chapter/chapter11}

% chapter12 Bio.PopGen: Population genetics
\chapter{Bio.PopGen: Population genetics}

Bio.PopGen is a Biopython module supporting population genetics,
available in Biopython 1.44 onwards.

The medium term objective for the module is to support widely used data
formats, applications and databases. This module is currently under intense
development and support for new features should appear at a rather fast pace.
Unfortunately this might also entail some instability on the API, especially
if you are using a development version. APIs that are made available on
our official public releases should be much more stable.

\section{GenePop}

GenePop (\url{http://genepop.curtin.edu.au/}) is a popular population
genetics software package supporting Hardy-Weinberg tests, linkage
desiquilibrium, population diferentiation, basic statistics, $F_{st}$ and
migration estimates, among others. GenePop does not supply sequence
based statistics as it doesn't handle sequence data.
The GenePop file format is supported by a wide range of other population
genetic software applications, thus making it a relevant format in the
population genetics field.

Bio.PopGen provides a parser and generator of GenePop file format.
Utilities to manipulate the content of a record are also provided.
Here is an example on how to read a GenePop file (you can find
example GenePop data files in the Test/PopGen directory of Biopython):

\begin{verbatim}
from Bio.PopGen import GenePop

handle = open("example.gen")
rec = GenePop.read(handle)
handle.close()
\end{verbatim}

This will read a file called example.gen and parse it. If you
do print rec, the record will be output again, in GenePop format.

The most important information in rec will be the loci names and
population information (but there is more -- use help(GenePop.Record)
to check the API documentation). Loci names can be found on rec.loci\_list.
Population information can be found on rec.populations.
Populations is a list with one element per population. Each element is itself
a list of individuals, each individual is a pair composed by individual
name and a list of alleles (2 per marker), here is an example for
rec.populations:

\begin{verbatim}
[
    [
        ('Ind1', [(1, 2),    (3, 3), (200, 201)],
        ('Ind2', [(2, None), (3, 3), (None, None)],
    ],
    [
        ('Other1', [(1, 1),  (4, 3), (200, 200)],
    ]
]
\end{verbatim}

So we have two populations, the first with two individuals, the
second with only one. The first individual of the first
population is called Ind1, allelic information for each of
the 3 loci follows. Please note that for any locus, information
might be missing (see as an example, Ind2 above).

A few utility functions to manipulate GenePop records are made
available, here is an example:

\begin{verbatim}
from Bio.PopGen import GenePop

#Imagine that you have loaded rec, as per the code snippet above...

rec.remove_population(pos)
#Removes a population from a record, pos is the population position in
#  rec.populations, remember that it starts on position 0.
#  rec is altered.

rec.remove_locus_by_position(pos)
#Removes a locus by its position, pos is the locus position in
#  rec.loci_list, remember that it starts on position 0.
#  rec is altered.

rec.remove_locus_by_name(name)
#Removes a locus by its name, name is the locus name as in
#  rec.loci_list. If the name doesn't exist the function fails
#  silently.
#  rec is altered.

rec_loci = rec.split_in_loci()
#Splits a record in loci, that is, for each loci, it creates a new
#  record, with a single loci and all populations.
#  The result is returned in a dictionary, being each key the locus name.
#  The value is the GenePop record.
#  rec is not altered.

rec_pops =  rec.split_in_pops(pop_names)
#Splits a record in populations, that is, for each population, it creates
#  a new record, with a single population and all loci.
#  The result is returned in a dictionary, being each key
#  the population name. As population names are not available in GenePop,
#  they are passed in array (pop_names).
#  The value of each dictionary entry is the GenePop record.
#  rec is not altered.
\end{verbatim}

GenePop does not support population names, a limitation which can be
cumbersome at times. Functionality to enable population names is currently
being planned for Biopython. These extensions won't break compatibility in
any way with the standard format.  In the medium term, we would also like to
support the GenePop web service.

\section{Coalescent simulation}

A coalescent simulation is a backward model of population genetics with relation to
time. A simulation of ancestry is done until the Most Recent Common Ancestor (MRCA) is found.
This ancestry relationship starting on the MRCA and ending on the current generation
sample is sometimes called a genealogy. Simple cases assume a population of constant
size in time, haploidy, no population structure, and simulate the alleles of a single
locus under no selection pressure.

Coalescent theory is used in many fields like selection detection, estimation of
demographic parameters of real populations or disease gene mapping.

The strategy followed in the Biopython implementation of the coalescent was not
to create a new, built-in, simulator from scratch but to use an existing one,
Fastsimcoal2 (\url{http://cmpg.unibe.ch/software/fastsimcoal2/}). Fastsimcoal2 allows for,
among others, population structure, multiple demographic events, simulation
of multiple types of loci (SNPs, sequences, STRs/microsatellites and RFLPs)
with recombination, diploidy multiple chromosomes or ascertainment bias. Notably
Fastsimcoal2 doesn't support any selection model. We recommend reading
Fastsimcoal2's
documentation, available in the link above.

The input for Fastsimcoal2 is a file specifying the desired demography and genome,
the output is a set of files (typically around 1000) with the simulated genomes
of a sample of individuals per subpopulation. This set of files can be used
in many ways, like to compute confidence intervals where which certain
statistics (e.g., $F_{st}$ or Tajima D) are expected to lie. Real population
genetics datasets statistics can then be compared to those confidence intervals.

Biopython coalescent code allows to create demographic scenarios and genomes and
to run Fastsimcoal2.

\subsection{Creating scenarios}

Creating a scenario involves both creating a demography and a chromosome structure.
In many cases (e.g. when doing Approximate Bayesian Computations -- ABC) it is
important to test many parameter variations (e.g. vary the effective population size,
$N_e$, between 10, 50, 500 and 1000 individuals). The code provided allows for
the simulation of scenarios with different demographic parameters very easily.

Below we see how we can create scenarios and then how simulate them.

\subsubsection{Demography}

A few predefined demographies are built-in, all have two shared parameters: sample size
(called sample\_size on the template, see below for its use) per deme and deme size, i.e.
subpopulation size (pop\_size). All demographies are available as templates where all
parameters can be varied, each template has a system name. The prefedined
demographies/templates are:

\begin{description}
\item[Single population, constant size] The standard parameters are enough to specify
it. Template name: simple.
\item[Single population, bottleneck] As seen on figure \ref{fig:bottle}. The parameters
are current population size (pop\_size on template ne3 on figure), time of expansion,
given as the generation in the past when it occurred (expand\_gen), 
effective population size during bottleneck (ne2), time of contraction
(contract\_gen) and original size in the remote past (ne3). Template name: bottle.
\item[Island model] The typical island model. The total number of demes is specified
by total\_demes and the migration rate by mig. Template name island.
\item[Stepping stone model - 1 dimension] The stepping stone model in 1 dimension,
extremes disconnected. The total number of demes is total\_demes, migration rate
is mig. Template name is ssm\_1d.
\item[Stepping stone model - 2 dimensions] The stepping stone model in 2 dimensions,
extremes disconnected. The parameters are x for the horizontal dimension and y
for the vertical (being the total number of demes x times y), migration rate is mig.
Template name is ssm\_2d.
\end{description}

\begin{htmlonly}
\label{fig:bottle}
\imgsrc{images/bottle.png}
\end{htmlonly}

\begin{latexonly}
\begin{figure}[htbp]
\centering
\includegraphics{images/bottle.png}
\caption{A bottleneck}
\label{fig:bottle}
\end{figure}
\end{latexonly}

In our first example, we will generate a template for a single population, constant size
model with a sample size of 30 and a deme size of 500. The code for this is:

\begin{verbatim}
from Bio.PopGen.SimCoal.Template import generate_simcoal_from_template

generate_simcoal_from_template('simple',
    [(1, [('SNP', [24, 0.0005, 0.0])])],
    [('sample_size', [30]),
    ('pop_size', [100])])
\end{verbatim}

Executing this code snippet will generate a file on the current directory called
simple\_100\_300.par this file can be given as input to Fastsimcoal2 to simulate the
demography (below we will see how Biopython can take care of calling
Fastsimcoal2).

This code consists of a single function call, let's discuss it parameter by parameter.

The first parameter is the template id (from the list above). We are using the id
'simple' which is the template for a single population of constant size along time.

The second parameter is the chromosome structure. Please ignore it for now, it will be
explained in the next section.

The third parameter is a list of all required parameters (recall that the simple model
only needs sample\_size and pop\_size) and possible values (in this case each
parameter only has a possible value).

Now, let's consider an example where we want to generate several island models, and we
are interested in varying the number of demes: 10, 50 and 100 with a migration
rate of 1\%. Sample size and deme
size will be the same as before. Here is the code:


\begin{verbatim}
from Bio.PopGen.SimCoal.Template import generate_simcoal_from_template

generate_simcoal_from_template('island',
    [(1, [('SNP', [24, 0.0005, 0.0])])],
    [('sample_size', [30]),
    ('pop_size', [100]),
    ('mig', [0.01]),
    ('total_demes', [10, 50, 100])])
\end{verbatim}

In this case, 3 files will be generated: island\_100\_0.01\_100\_30.par,
island\_10\_0.01\_100\_30.par and island\_50\_0.01\_100\_30.par. Notice the
rule to make file names: template name, followed by parameter values in
reverse order.

A few, arguably more esoteric template demographies exist (please check the
Bio/PopGen/SimCoal/data directory on Biopython source tree). Furthermore it is possible
for the user to create new templates. That functionality will be discussed in a future
version of this document.

\subsubsection{Chromosome structure}

We strongly recommend reading Fastsimcoal2 documentation to understand the full potential
available in modeling chromosome structures. In this subsection we only discuss how
to implement chromosome structures using the Biopython interface, not the underlying
Fastsimcoal2 capabilities.

We will start by implementing a single chromosome, with 24 SNPs with
a recombination rate immediately on the right of each locus of 0.0005 and a
minimum frequency of the minor allele of 0. This will be specified by the
following list (to be passed as second parameter to the function
generate\_simcoal\_from\_template):

\begin{verbatim}
[(1, [('SNP', [24, 0.0005, 0.0])])]
\end{verbatim}

This is actually the chromosome structure used in the above examples.


The chromosome structure is represented by a list of chromosomes,
each chromosome (i.e., each element in the list)
is composed by a tuple (a pair): the first element
is the number of times the chromosome is to be repeated (as there
might be interest in repeating the same chromosome many times).
The second element is a list of the actual components of the chromosome.
Each element is again a pair, the first member is the locus type and
the second element the parameters for that locus type. Confused?
Before showing more examples let's review the example above: We have
a list with one element (thus one chromosome), the chromosome is
a single instance (therefore not to be repeated), it is composed
of 24 SNPs, with a recombination rate of 0.0005 between each
consecutive SNP, the minimum frequency of the minor allele is
0.0 (i.e, it can be absent from a certain population).

Let's see a more complicated example:

\begin{verbatim}
[
  (5, [
       ('SNP', [24, 0.0005, 0.0])
      ]
  ),
  (2, [
       ('DNA', [10, 0.0, 0.00005, 0.33]),
       ('RFLP', [1, 0.0, 0.0001]),
       ('MICROSAT', [1, 0.0, 0.001, 0.0, 0.0])
      ]
  )
]
\end{verbatim}

We start by having 5 chromosomes with the same structure as
above (i.e., 24 SNPs). We then have 2 chromosomes which
have a DNA sequence with 10 nucleotides, 0.0 recombination rate,
0.0005 mutation rate, and a transition rate of 0.33. Then we
have an RFLP with 0.0 recombination rate to the next locus and
a 0.0001 mutation rate. Finally we have a microsatellite (or STR),
with 0.0 recombination rate to the next locus (note, that as this
is a single microsatellite which has no loci following, this
recombination rate here is irrelevant), with a mutation rate
of 0.001, geometric parameter of 0.0 and a range constraint
of 0.0 (for information about this parameters please consult
the Fastsimcoal2 documentation, you can use them to simulate
various mutation models, including the typical  -- for microsatellites --
stepwise mutation model among others).


\subsection{Running Fastsimcoal2}

We now discuss how to run Fastsimcoal2 from inside Biopython. It is required
that the binary for Fastsimcoal2 is called fastsimcoal21 (or fastsimcoal21.exe on Windows
based platforms), please note that the typical name when downloading the
program is in the format fastsimcoal2\_x\_y. As such, when installing
Fastsimcoal2
you will need to rename of the downloaded executable so that Biopython can
find it.

It is possible to run Fastsimcoal2 on files that were not generated using the method
above (e.g., writing a parameter file by hand), but we will show an
example by creating a model using the framework presented above.

\begin{verbatim}
from Bio.PopGen.SimCoal.Template import generate_simcoal_from_template
from Bio.PopGen.SimCoal.Controller import FastSimCoalController


generate_simcoal_from_template('simple',
    [
      (5, [
           ('SNP', [24, 0.0005, 0.0])
          ]
      ),
      (2, [
           ('DNA', [10, 0.0, 0.00005, 0.33]),
           ('RFLP', [1, 0.0, 0.0001]),
           ('MICROSAT', [1, 0.0, 0.001, 0.0, 0.0])
          ]
      )
    ],
    [('sample_size', [30]),
    ('pop_size', [100])])

ctrl = FastSimCoalController()
ctrl.run_fastsimcoal('simple_100_30.par', 50)
\end{verbatim}

The lines of interest are the last two (plus the new import).
Firstly a controller for the
application is created. The directory where the binary is located has
to be specified.

The simulator is then run on the last line: we know, from the rules explained
above, that the input file name is simple\_100\_30.par for the
simulation parameter file created. We then specify
that we want to run 50 independent simulations, by default Biopython
requests a simulation of diploid data, but a third parameter can
be added to simulate haploid data (adding as a parameter the
string '0'). Fastsimcoal2 will now run (please
note that this can take quite a lot of time) and will create a directory
with the simulation results. The results can now be analysed (typically
studying the data with Arlequin3). In the future Biopython might support
reading the Arlequin3 format and thus allowing for the analysis of
Fastsimcoal2
data inside Biopython.


\section{Other applications}

Here we discuss interfaces and utilities to deal with population genetics'
applications which arguably have a smaller user base.

\subsection{FDist: Detecting selection and molecular adaptation}

FDist is a selection detection application suite based on computing
(i.e. simulating) a ``neutral'' confidence interval based on $F_{st}$ and
heterozygosity. Markers (which can be SNPs, microsatellites, AFLPs
among others) which lie outside the ``neutral'' interval are to be
considered as possible candidates for being under selection.

FDist is mainly used when the number of markers is considered enough
to estimate an average $F_{st}$, but not enough to either have outliers
calculated from the dataset directly or, with even more markers for
which the relative positions in the genome are known, to use
approaches based on, e.g., Extended Haplotype Heterozygosity (EHH).

The typical usage pattern for FDist is as follows:

\begin{enumerate}
\item Import a dataset from an external format into FDist format.
\item Compute average $F_{st}$. This is done by datacal inside FDist.
\item Simulate ``neutral'' markers based on the
    average $F_{st}$ and expected number of total populations.
    This is the core operation, done by fdist inside FDist.
\item Calculate the confidence interval, based on the desired
    confidence boundaries (typically 95\% or 99\%). This is done by
    cplot and is mainly used to plot the interval.
\item Assess each marker status against the simulation ``neutral''
    confidence interval. Done
    by pv. This is used to detect the outlier status of each marker
    against the simulation.
\end{enumerate}

We will now discuss each step with illustrating example code
(for this example to work FDist binaries have to be on the
executable PATH).

The FDist data format is application specific and is not used at
all by other applications, as such you will probably have to convert
your data for use with FDist. Biopython can help you do this.
Here is an example converting from GenePop format to FDist format
(along with imports that will be needed on examples further below):

\begin{verbatim}
from Bio.PopGen import GenePop
from Bio.PopGen import FDist
from Bio.PopGen.FDist import Controller
from Bio.PopGen.FDist.Utils import convert_genepop_to_fdist

gp_rec = GenePop.read(open("example.gen"))
fd_rec = convert_genepop_to_fdist(gp_rec)
in_file = open("infile", "w")
in_file.write(str(fd_rec))
in_file.close()
\end{verbatim}

In this code we simply parse a GenePop file and convert it to a FDist
record.

Printing an FDist record will generate
a string that can be directly saved to a file and supplied to FDist. FDist
requires the input file to be called infile, therefore we save the record on
a file with that name.

The most important fields on a FDist record are: num\_pops, the number of
populations; num\_loci, the number of loci and loci\_data with the marker
data itself. Most probably the details of the record are of no interest
to the user, as the record only purpose is to be passed to FDist.

The next step is to calculate the average $F_{st}$ of the dataset (along
with the sample size):

\begin{verbatim}
ctrl = Controller.FDistController()
fst, samp_size = ctrl.run_datacal()
\end{verbatim}

On the first line we create an object to control the call of  FDist
suite, this object will be used further on in order to call other
suite applications.

On the second line we call the datacal application which computes the
average $F_{st}$
and the sample size. It is worth noting that the $F_{st}$ computed by
datacal is a \emph{variation} of Weir and Cockerham's $\theta$.

We can now call the main fdist application in order to simulate neutral
markers.

\begin{verbatim}
sim_fst = ctrl.run_fdist(npops = 15, nsamples = fd_rec.num_pops, fst = fst,
    sample_size = samp_size, mut = 0, num_sims = 40000)
\end{verbatim}

\begin{description}
\item[npops] Number of populations existing in nature. This is really a
    ``guestimate''. Has to be lower than 100.
\item[nsamples] Number of populations sampled, has to be lower than npops.
\item[fst] Average $F_{st}$.
\item[sample\_size] Average number of individuals sampled on each population.
\item[mut] Mutation model: 0 - Infinite alleles; 1 - Stepwise mutations
\item[num\_sims] Number of simulations to perform. Typically a number around
    40000 will be OK, but if you get a confidence interval that looks sharp
    (this can be detected when plotting the confidence interval computed
    below) the value can be increased (a suggestion would be steps of 10000
    simulations).
\end{description}

The confusion in wording between number of samples and sample size
stems from the original application.

A file named out.dat will be created with the simulated heterozygosities
and $F_{st}$s, it will have as many lines as the number of simulations
requested.

Note that fdist returns the average $F_{st}$ that it was \emph{capable} of
simulating, for more details about this issue please read below the paragraph
on approximating the desired average $F_{st}$.

The next (optional) step is to calculate the confidence interval:

\begin{verbatim}
cpl_interval = ctrl.run_cplot(ci=0.99)
\end{verbatim}

You can only call cplot after having run fdist.

This will calculate the confidence intervals (99\% in this case)
for a previous fdist run. A list of quadruples is returned. The
first element represents the heterozygosity, the second the lower
bound of $F_{st}$ confidence interval for that heterozygosity,
the third the average and the fourth the upper bound. This can
be used to trace the confidence interval contour. This list
is also written to a file, out.cpl.

The main purpose of this step is return a set of points which can
be easily used to plot a confidence interval. It can be skipped
if the objective is only to assess the status of each marker against
the simulation, which is the next step...

\begin{verbatim}
pv_data = ctrl.run_pv()
\end{verbatim}

You can only call cplot after having run datacal and fdist.

This will use the simulated markers to assess the status of each
individual real marker. A list, in the same order than the loci\_list
that is on the FDist record (which is in the same order that the GenePop
record) is returned. Each element in the list is a quadruple, the
fundamental member of each quadruple is the last element (regarding the
other elements, please refer to the pv documentation -- for the
sake of simplicity we will not discuss them here) which returns the
probability of the simulated $F_{st}$ being lower than the marker $F_{st}$.
Higher values would indicate a stronger candidate for positive selection,
lower values a candidate for balancing selection, and intermediate values
a possible neutral marker. What is ``higher'', ``lower'' or ``intermediate''
is really a subjective issue, but taking a ``confidence interval'' approach
and considering a 95\% confidence interval, ``higher'' would be between 0.95
and 1.0, ``lower'' between 0.0 and 0.05 and ``intermediate'' between 0.05 and
0.95.

\subsubsection{Approximating the desired average $F_{st}$}

Fdist tries to approximate the desired average $F_{st}$ by doing a
coalescent simulation using migration rates based on the formula

\[ N_{m} = \frac{1 - F_{st}}{4F_{st}} \]

This formula assumes a few premises like an infinite number of populations.

In practice, when the number of populations is low, the mutation model
is stepwise and the sample size increases, fdist will not be able to
simulate an acceptable approximate average $F_{st}$.

To address that, a function is provided to iteratively approach the desired
value by running several fdists in sequence. This approach is computationally
more intensive than running a single fdist run, but yields good results.
The following code runs fdist approximating the desired $F_{st}$:

\begin{verbatim}
sim_fst = ctrl.run_fdist_force_fst(npops = 15, nsamples = fd_rec.num_pops,
    fst = fst, sample_size = samp_size, mut = 0, num_sims = 40000,
    limit = 0.05)
\end{verbatim}

The only new optional parameter, when comparing with run\_fdist, is limit
which is the desired maximum error. run\_fdist can (and probably should)
be safely replaced with run\_fdist\_force\_fst.

\subsubsection{Final notes}

The process to determine the average $F_{st}$ can be more sophisticated than
the one presented here. For more information we refer you to the FDist
README file. Biopython's code can be used to implement more sophisticated
approaches.

\section{Future Developments}

The most desired future developments would be the ones you add yourself ;) .

That being said, already existing fully functional code is currently being
incorporated in Bio.PopGen, that code covers the applications FDist and
SimCoal2, the HapMap and UCSC Table Browser databases and some simple statistics
like $F_{st}$, or allele counts.


% chapter13 Phylogenetics with Bio.Phylo
\include{chapter/chapter13}

% chapter14 Sequence motif analysis using Bio.motifs
\include{chapter/chapter14}

% chapter15 Cluster analysis
\include{chapter/chapter15}

% chapter16 Supervised learning methods
\include{chapter/chapter16}

% chapter17 Graphics including GenomeDiagram
\include{chapter/chapter17}

% chapter18 Codon Alignment
\chapter{Codon Alignment (\textit{experimental code})}
\label{chap:codonalignment}

WARNING: This chapter of the Tutorial describes an experimental module in
Biopython. It is being included in Biopython and documented here in the
tutorial in a pre-final state to allow a period of feedback and refinement
before we declare it stable. Until then the details will probably change,
and any scripts using the current Bio.SearchIO would need to be updated.
Please keep this in mind!

This chapter is about Codon Alignment, which is a special case of
nucleotide alignment in which the trinucleotides correspond directly to
amino acids in the translated protein product. Codon Alignment carries
information that can be used for a variety of evolutionary analysis.

This chapter has been divided into four parts to explain the codon
alignment support in Biopython. First, a general introduction about the
basic classes in \verb|Bio.CodonAlign| will be given. Then, a typical
procedure of how to obtain a codon alignment within Biopython is
discussed. Next, some simple applications of codon alignment, such as
dN/dS ratio estimation and neutrality test and so forth will be covered.
Finally, IO support of codon alignment will help user to conduct
analysis that cannot be done within Biopython.

\section{CodonSeq Class}

\verb|Bio.CodonAlign.CodonSeq| class is the base class in Codon
Alignment. It is similar to \verb|Bio.Seq| but with some extra
attributes. To obtain a \verb|CodonSeq| object, you just need
to give a \verb|str| object of nucleotide sequence whose length is a
multiple of 3 (This can be violated if you have \texttt{rf\_table}
argument). For example:

%doctest
\begin{verbatim}
>>> from Bio.CodonAlign import CodonSeq
>>> codon_seq = CodonSeq("AAATTTCCCGGG")
>>> codon_seq
CodonSeq('AAATTTCCCGGG', CodonAlphabet(Standard))
\end{verbatim}

An error will raise up if the input sequence is not a multiple of 3.

\begin{verbatim}
>>> codon_seq = CodonSeq("AAATTTCCCGG")
Traceback (most recent call last):
  File "<stdin>", line 1, in <module>
  File "/biopython/Bio/CodonAlign/CodonSeq.py", line 81, in __init__
    assert len(self) % 3 == 0, "Sequence length is not a triple number"
AssertionError: Sequence length is not a triple number
\end{verbatim}

By default, \verb|Bio.CodonAlign.default_codon_alphabet| will be
assigned to \verb|CodonSeq| object if you don't specify any Alphabet.
This \verb|default_codon_alphabet| is gapped universal genetic code,
which will work in most cases. However, if you are analyzing data from
mitochondria, for instance, and are in need of assigning an special
codon alphabet by yourself, \verb|Bio.CodonAlign.CodonAlphabet| also
provides you an easy solution. All you need is to pick up a
\verb|CodonTable| object that is correct for your data. For example:

%doctest
\begin{verbatim}
>>> from Bio.CodonAlign import CodonSeq
>>> from Bio.CodonAlign.CodonAlphabet import get_codon_alphabet
>>> from Bio.Data.CodonTable import generic_by_id    # vertebrate mitochondria alphabet
>>> codon_alphabet = get_codon_alphabet(generic_by_id[2], gap_char="-")
>>> codon_seq1 = CodonSeq("AAA---CCCGGG", alphabet=codon_alphabet)
>>> codon_seq1
CodonSeq('AAA---CCCGGG', CodonAlphabet(Vertebrate Mitochondrial))
\end{verbatim}

The slice of \verb|CodonSeq| is exactly the same with \verb|Seq| and
it will always return a \verb|Seq| object if you sliced a
\verb|CodonSeq|. For example:

%cont-doctest
\begin{verbatim}
>>> codon_seq1
CodonSeq('AAA---CCCGGG', CodonAlphabet(Vertebrate Mitochondrial))
>>> codon_seq1[:6]
Seq('AAA---', DNAAlphabet())
>>> codon_seq1[1:5]
Seq('AA--', DNAAlphabet())
\end{verbatim}

As you might imagine, \verb|CodonSeq| is able to be translated into
amino acid sequence based on the \verb|CodonAlphabet| within it. In
fact, \verb|CodonSeq| does more than this. \verb|CodonSeq| object
has a \texttt{rf\_table} attribute that dictates how the
\verb|CodonSeq| will be translated (\texttt{rf\_table} will indicate
the starting position of each codon in the sequence). This is useful if
your sequence is known to have frameshift events (genomic DNA) or
pseudogene that has insertion or deletion. You might notice that in the
previous example, you haven't specify the \texttt{rf\_table} when initiate
a \verb|CodonSeq| object. In fact, \verb|CodonSeq| object will
automatically assign a \texttt{rf\_table} to the \texttt{CodonSeq} if
you don't say anything about it.


In the example, we didn't assign \texttt{rf\_table} to
\texttt{codon\_seq1}. By default, \verb|CodonSeq| will automatically
generate a \texttt{rf\_table} to the coding sequence assuming no
frameshift events. In this case, it is \texttt{{[}0, 3, 6{]}}, which
means the first codon in the sequence starts at position 0, the second
codon in the sequence starts at position 3, and the third codon in the
sequence starts at position 6. In \texttt{codon\_seq2}, we only have 8
nucleotides in the sequence, but with \texttt{rf\_table} option
specified. In this case, the third codon starts at the 5th position of
the sequence rather than the 6th. And the \texttt{translate()} function
will use the \texttt{rf\_table} to get the translated amino acid
sequence.

Another thing to keep in mind is that \texttt{rf\_table} will only be
applied to ungapped nucleotide sequence. This makes \texttt{rf\_table}
to be interchangeable between \verb|CodonSeq| with the same sequence
but different gaps inserted. For example,

%cont-doctest
\begin{verbatim}
>>> codon_seq1 = CodonSeq("AAACCC---GGG")
>>> codon_seq1.rf_table
[0, 3, 6]
>>> codon_seq1.translate()
'KPG'
>>> codon_seq1.full_translate()
'KP-G'
\end{verbatim}

We can see that the \texttt{rf\_table} of \texttt{codon\_seq1} is still
\texttt{{[}0, 3, 6{]}}, even though we have gaps added. The
\verb|translate()| function will skip the gaps and return the ungapped
amino acid sequence. If gapped protein sequence is what you need,
\verb|full_translate()| comes to help.

It is also easy to convert \verb|Seq| object to \verb|CodonSeq|
object, but it is the user's responsibility to ensure all the necessary
information is correct for a \verb|CodonSeq| (mainly \texttt{rf\_table}).

\begin{verbatim}
>>> from Bio.Seq import Seq
>>> codon_seq = CodonSeq()
>>> seq = Seq('AAAAAA')
>>> codon_seq.from_seq(seq)
CodonSeq('AAAAAA', CodonAlphabet(Standard))
>>> seq = Seq('AAAAA')
>>> codon_seq.from_seq(seq)
Traceback (most recent call last):
  File "<stdin>", line 1, in <module>
  File "/biopython/Bio/CodonAlign/CodonSeq.py", line 264, in from_seq
    return cls(seq._data, alphabet=alphabet)
  File "/biopython/Bio/CodonAlign/CodonSeq.py", line 80, in __init__
    assert len(self) % 3 == 0, "Sequence length is not a triple number"
AssertionError: Sequence length is not a triple number
>>> codon_seq.from_seq(seq, rf_table=(0, 2))
CodonSeq('AAAAA', CodonAlphabet(Standard))
\end{verbatim}

\section{CodonAlignment Class}

The \verb|CodonAlignment| class is another new class in
\verb|Codon.Align|. It's aim is to store codon alignment data and
apply various analysis upon it. Similar to
\verb|MultipleSeqAlignment|, you can use numpy style slice to a
\verb|CodonAlignment|. However, once you sliced, the returned result
will always be a \verb|MultipleSeqAlignment| object.

%doctest
\begin{verbatim}
>>> from Bio.CodonAlign import default_codon_alphabet, CodonSeq, CodonAlignment
>>> from Bio.Alphabet import generic_dna
>>> from Bio.SeqRecord import SeqRecord
>>> from Bio.Alphabet import IUPAC, Gapped
>>> a = SeqRecord(CodonSeq("AAAACGTCG", alphabet=default_codon_alphabet), id="Alpha")
>>> b = SeqRecord(CodonSeq("AAA---TCG", alphabet=default_codon_alphabet), id="Beta")
>>> c = SeqRecord(CodonSeq("AAAAGGTGG", alphabet=default_codon_alphabet), id="Gamma")
>>> codon_aln = CodonAlignment([a, b, c])
>>> print(codon_aln)
CodonAlphabet(Standard) CodonAlignment with 3 rows and 9 columns (3 codons)
AAAACGTCG Alpha
AAA---TCG Beta
AAAAGGTGG Gamma
>>> print(codon_aln[0])
ID: Alpha
Name: <unknown name>
Description: <unknown description>
Number of features: 0
CodonSeq('AAAACGTCG', CodonAlphabet(Standard))
>>> print(codon_aln[:, 3])
A-A
>>> print(codon_aln[1:, 3:10])
NucleotideAlphabet() alignment with 2 rows and 6 columns
---TCG Beta
AGGTGG Gamma
\end{verbatim}

You can write out \verb|CodonAlignment| object just as what you do
with \verb|MultipleSeqAlignment|.

\begin{verbatim}
>>> from Bio import AlignIO
>>> AlignIO.write(codon_aln, 'example.aln', 'clustal')
\end{verbatim}

An alignment file called \texttt{example.aln} can then be found in your
current working directory. You can write \verb|CodonAlignment| out in
any MSA format that Biopython supports.

Currently, you are not able to read MSA data as a
\verb|CodonAlignment| object directly (because of dealing with
\texttt{rf\_table} issue for each sequence). However, you can read the
alignment data in as a \verb|MultipleSeqAlignment| object and convert
them into \verb|CodonAlignment| object using \verb|from_msa()|
class method. For example,

\begin{verbatim}
>>> aln = AlignIO.read('example.aln', 'clustal')
>>> codon_aln = CodonAlignment()
>>> print(codon_aln.from_msa(aln))
CodonAlphabet(Standard) CodonAlignment with 3 rows and 9 columns (3 codons)
AAAACGTCG Alpha
AAA---TCG Beta
AAAAGGTGG Gamma
\end{verbatim}

Note, the \verb|from_msa()| method assume there is no frameshift
events occurs in your alignment. Its behavior is not guaranteed if your
sequence contain frameshift events!!

There is a couple of methods that can be applied to
\verb|CodonAlignment| class for evolutionary analysis. We will cover
them more in \ref{subsec:app}.

\section{Build a Codon Alignment}

Building a codon alignment is the first step of many evolutionary
anaysis. But how to do that? \verb|Bio.CodonAlign| provides you an
easy funciton --- \verb|build()| to achieve all. The data you need to
prepare in advance is a protein alignment and a set of DNA sequences
that can be translated into the protein sequences in the alignment.

\verb|CodonAlign.build()| method requires two mandatory arguments. The
first one should be a protein \verb|MultipleSeqAlignment| object and
the second one is a list of nucleotide \verb|SeqRecord| object. By
default, \verb|CodonAlign.build()| assumes the order of the alignment
and nucleotide sequences are in the same. For example:

%doctest
\begin{verbatim}
>>> from Bio import CodonAlign
>>> from Bio.Alphabet import IUPAC
>>> from Bio.Align import MultipleSeqAlignment
>>> from Bio.SeqRecord import SeqRecord
>>> from Bio.Seq import Seq
>>> nucl1 = SeqRecord(Seq('AAATTTCCCGGG', alphabet=IUPAC.IUPACUnambiguousDNA()), id='nucl1')
>>> nucl2 = SeqRecord(Seq('AAATTACCCGCG', alphabet=IUPAC.IUPACUnambiguousDNA()), id='nucl2')
>>> nucl3 = SeqRecord(Seq('ATATTACCCGGG', alphabet=IUPAC.IUPACUnambiguousDNA()), id='nucl3')
>>> prot1 = SeqRecord(nucl1.seq.translate(), id='prot1')
>>> prot2 = SeqRecord(nucl2.seq.translate(), id='prot2')
>>> prot3 = SeqRecord(nucl3.seq.translate(), id='prot3')
>>> aln = MultipleSeqAlignment([prot1, prot2, prot3])
>>> codon_aln = CodonAlign.build(aln, [nucl1, nucl2, nucl3])
>>> print(codon_aln)
CodonAlphabet(Standard) CodonAlignment with 3 rows and 12 columns (4 codons)
AAATTTCCCGGG nucl1
AAATTACCCGCG nucl2
ATATTACCCGGG nucl3
\end{verbatim}

In the above example, \verb|CodonAlign.build()| will try to match
\texttt{nucl1} with \texttt{prot1}, \texttt{nucl2} with \texttt{prot2}
and \texttt{nucl3} with \texttt{prot3}, i.e., assuming the order of
records in \texttt{aln} and \texttt{{[}nucl1, nucl2, nucl3{]}} is the
same.

\verb|CodonAlign.build()| method is also able to handle key match. In
this case, records with same id are paired. For example:

%cont-doctest
\begin{verbatim}
>>> nucl1 = SeqRecord(Seq('AAATTTCCCGGG', alphabet=IUPAC.IUPACUnambiguousDNA()), id='nucl1')
>>> nucl2 = SeqRecord(Seq('AAATTACCCGCG', alphabet=IUPAC.IUPACUnambiguousDNA()), id='nucl2')
>>> nucl3 = SeqRecord(Seq('ATATTACCCGGG', alphabet=IUPAC.IUPACUnambiguousDNA()), id='nucl3')
>>> prot1 = SeqRecord(nucl1.seq.translate(), id='prot1')
>>> prot2 = SeqRecord(nucl2.seq.translate(), id='prot2')
>>> prot3 = SeqRecord(nucl3.seq.translate(), id='prot3')
>>> aln = MultipleSeqAlignment([prot1, prot2, prot3])
>>> nucl = {'prot1': nucl1, 'prot2': nucl2, 'prot3': nucl3}
>>> codon_aln = CodonAlign.build(aln, nucl)
>>> print(codon_aln)
CodonAlphabet(Standard) CodonAlignment with 3 rows and 12 columns (4 codons)
AAATTTCCCGGG nucl1
AAATTACCCGCG nucl2
ATATTACCCGGG nucl3
\end{verbatim}

This option is handleful if you read nucleotide sequences using
\verb|SeqIO.index()| method, in which case the nucleotide dict with be
generated automatically.

Sometimes, you are neither not able to ensure the same order or the same
id. \verb|CodonAlign.build()| method provides you an manual approach to
tell the program nucleotide sequence and protein sequence correspondance
by generating a \texttt{corr\_dict}. \texttt{corr\_dict} should be a
dictionary that uses protein record id as key and nucleotide record id
as item. Let's look at an example:

%cont-doctest
\begin{verbatim}
>>> nucl1 = SeqRecord(Seq('AAATTTCCCGGG', alphabet=IUPAC.IUPACUnambiguousDNA()), id='nucl1')
>>> nucl2 = SeqRecord(Seq('AAATTACCCGCG', alphabet=IUPAC.IUPACUnambiguousDNA()), id='nucl2')
>>> nucl3 = SeqRecord(Seq('ATATTACCCGGG', alphabet=IUPAC.IUPACUnambiguousDNA()), id='nucl3')
>>> prot1 = SeqRecord(nucl1.seq.translate(), id='prot1')
>>> prot2 = SeqRecord(nucl2.seq.translate(), id='prot2')
>>> prot3 = SeqRecord(nucl3.seq.translate(), id='prot3')
>>> aln = MultipleSeqAlignment([prot1, prot2, prot3])
>>> corr_dict = {'prot1': 'nucl1', 'prot2': 'nucl2', 'prot3': 'nucl3'}
>>> codon_aln = CodonAlign.build(aln, [nucl3, nucl1, nucl2], corr_dict=corr_dict)
>>> print(codon_aln)
CodonAlphabet(Standard) CodonAlignment with 3 rows and 12 columns (4 codons)
AAATTTCCCGGG nucl1
AAATTACCCGCG nucl2
ATATTACCCGGG nucl3
\end{verbatim}

We can see, even though the second argument of \verb|CodonAlign.build()|
is not in the same order with \texttt{aln} in the above example, the
\texttt{corr\_dict} tells the program to pair protein records and
nucleotide records. And we are still able to obtain the correct
\verb|CodonAlignment| object.

The underlying algorithm of \verb|CodonAlign.build()| method is very
similar to \texttt{pal2nal} (a very famous perl script to build codon
alignment). \verb|CodonAlign.build()| will first translate protein
sequences into a long degenerate regular expression and tries to find a
match in its corresponding nucleotide sequence. When translation fails,
it divide protein sequence into several small anchors and tries to match
each anchor to the nucleotide sequence to figure out where the mismatch
and frameshift events lie. Other options available for
\verb|CodonAlign.build()| includes \texttt{anchor\_len} (default 10) and
\texttt{max\_score} (maximum tolerance of unexpected events, default
10). You may want to refer the Biopython build-in help to get more
information about these options.

Now let's look at a real example of building codon alignment. Here we
will use epidermal growth factor (EGFR) gene to demonstrate how to
obtain codon alignment. To reduce your effort, we have already collected
EGFR sequences for Homo sapiens, Bos taurus, Rattus norvegicus,
Sus scrofa and Drosophila melanogaster. You can download them from
\href{http://zruanweb.com/egfr.zip}{here}. Uncomressing the
\texttt{.zip}, you will see three files. \texttt{egfr\_nucl.fa} is
nucleotide sequences of EGFR and \texttt{egfr\_pro.aln} is EGFR protein
sequence alignment in \texttt{clustal} format. The \texttt{egfr\_id}
contains id correspondance between protein records and nucleotide
records. You can then try the following code (make sure the files are in
your current python working directory):

\begin{verbatim}
>>> from Bio import SeqIO, AlignIO
>>> nucl = SeqIO.parse('egfr_nucl.fa', 'fasta', alphabet=IUPAC.IUPACUnambiguousDNA())
>>> prot = AlignIO.read('egfr_pro.aln', 'clustal', alphabet=IUPAC.protein)
>>> id_corr = {i.split()[0]: i.split()[1] for i in open('egfr_id').readlines()}
>>> aln = CodonAlign.build(prot, nucl, corr_dict=id_corr, alphabet=CodonAlign.default_codon_alphabet)
/biopython/Bio/CodonAlign/__init__.py:568: UserWarning: gi|47522840|ref|NP_999172.1|(L 449) does not correspond to gi|47522839|ref|NM_214007.1|(ATG)
  % (pro.id, aa, aa_num, nucl.id, this_codon))
>>> print(aln)
CodonAlphabet(Standard) CodonAlignment with 6 rows and 4446 columns (1482 codons)
ATGATGATTATCAGCATGTGGATGAGCATATCGCGAGGATTGTGGGACAGCAGCTCC...GTG gi|24657088|ref|NM_057410.3|
---------------------ATGCTGCTGCGACGGCGCAACGGCCCCTGCCCCTTC...GTG gi|24657104|ref|NM_057411.3|
------------------------------ATGAAAAAGCACGAG------------...GCC gi|302179500|gb|HM749883.1|
------------------------------ATGCGACGCTCCTGGGCGGGCGGCGCC...GCA gi|47522839|ref|NM_214007.1|
------------------------------ATGCGACCCTCCGGGACGGCCGGGGCA...GCA gi|41327737|ref|NM_005228.3|
------------------------------ATGCGACCCTCAGGGACTGCGAGAACC...GCA gi|6478867|gb|M37394.2|RATEGFR
\end{verbatim}

We can see, while building the codon alignment a mismatch event is
found. And this is shown as a UserWarning.

\section{Codon Alignment Application}
\label{subsec:app}

The most important application of codon alignment is to estimate
nonsynonymous substitutions per site (dN) and synonymous substitutions
per site (dS). \verb|CodonAlign| currently support three counting
based methods (NG86, LWL85, YN00) and maximum likelihood method to
estimate dN and dS. The function to conduct dN, dS estimation is called
\verb|cal_dn_ds()|. When you obtained a codon alignment, it is quite
easy to calculate dN and dS. For example (assuming you have EGFR codon
alignmnet in the python working space):

\begin{verbatim}
>>> from Bio.CodonAlign.CodonSeq import cal_dn_ds
>>> print(aln)
CodonAlphabet(Standard) CodonAlignment with 6 rows and 4446 columns (1482 codons)
ATGATGATTATCAGCATGTGGATGAGCATATCGCGAGGATTGTGGGACAGCAGCTCC...GTG gi|24657088|ref|NM_057410.3|
---------------------ATGCTGCTGCGACGGCGCAACGGCCCCTGCCCCTTC...GTG gi|24657104|ref|NM_057411.3|
------------------------------ATGAAAAAGCACGAG------------...GCC gi|302179500|gb|HM749883.1|
------------------------------ATGCGACGCTCCTGGGCGGGCGGCGCC...GCA gi|47522839|ref|NM_214007.1|
------------------------------ATGCGACCCTCCGGGACGGCCGGGGCA...GCA gi|41327737|ref|NM_005228.3|
------------------------------ATGCGACCCTCAGGGACTGCGAGAACC...GCA gi|6478867|gb|M37394.2|RATEGFR
>>> dN, dS = cal_dn_ds(aln[0], aln[1], method='NG86')
>>> print(dN, dS)
0.0209078305058 0.0178371876389
>>> dN, dS = cal_dn_ds(aln[0], aln[1], method='LWL95')
>>> print(dN, dS)
0.0203061425453 0.0163935691992
>>> dN, dS = cal_dn_ds(aln[0], aln[1], method='YN00')
>>> print(dN, dS)
0.0198195580321 0.0221560648799
>>> dN, dS = cal_dn_ds(aln[0], aln[1], method='ML')
>>> print(dN, dS)
0.0193877676103 0.0217247139962
\end{verbatim}

If you are using maximum likelihood methdo to estimate dN and dS, you
are also able to specify equilibrium codon frequency to \texttt{cfreq}
argument. Available options include \texttt{F1x4}, \texttt{F3x4} and
\texttt{F61}.

It is also possible to get dN and dS matrix or a tree from a
\verb|CodonAlignment| object.

\begin{verbatim}
>>> dn_matrix, ds_matrix = aln.get_dn_ds_matrix()
>>> print(dn_matrix)
gi|24657088|ref|NM_057410.3|    0
gi|24657104|ref|NM_057411.3|    0.0209078305058 0
gi|302179500|gb|HM749883.1|     0.611523924924  0.61022032668   0
gi|47522839|ref|NM_214007.1|    0.614035083563  0.60401686212   0.0411803504059 0
gi|41327737|ref|NM_005228.3|    0.61415325314   0.60182631356   0.0670105144563 0.0614703609541 0
gi|6478867|gb|M37394.2|RATEGFR  0.61870883409   0.606868724887  0.0738690303483 0.0735789092792 0.0517984707257 0
gi|24657088|ref|NM_057410.3|    gi|24657104|ref|NM_057411.3|    gi|302179500|gb|HM749883.1| gi|47522839|ref|NM_214007.1|    gi|41327737|ref|NM_005228.3|    gi|6478867|gb|M37394.2|RATEGFR
>>> dn_tree, ds_tree = aln.get_dn_ds_tree()
>>> print(dn_tree)
Tree(rooted=True)
    Clade(branch_length=0, name='Inner5')
        Clade(branch_length=0.279185347322, name='Inner4')
            Clade(branch_length=0.00859186651689, name='Inner3')
                Clade(branch_length=0.0258992353629, name='gi|6478867|gb|M37394.2|RATEGFR')
                Clade(branch_length=0.0258992353629, name='gi|41327737|ref|NM_005228.3|')
            Clade(branch_length=0.0139009266768, name='Inner2')
                Clade(branch_length=0.020590175203, name='gi|47522839|ref|NM_214007.1|')
                Clade(branch_length=0.020590175203, name='gi|302179500|gb|HM749883.1|')
        Clade(branch_length=0.294630667432, name='Inner1')
            Clade(branch_length=0.0104539152529, name='gi|24657104|ref|NM_057411.3|')
            Clade(branch_length=0.0104539152529, name='gi|24657088|ref|NM_057410.3|')
\end{verbatim}

Another application of codon alignment that \verb|CodonAlign| supports
is Mcdonald-Kreitman test. This test compares the within species
synonymous substitutions and nonsynonymous substitutions and between
species synonymous substitutions and nonsynonymous substitutions to see
if they are from the same evolutionary process. The test requires gene
sequences sampled from different individuals of the same species. In the
following example, we will use Adh gene from fluit fly to demonstrate
how to conduct the test. The data includes 11 individuals from
D. melanogaster, 4 individuals from D. simulans and 12 individuals from
D. yakuba. The data is available from
\href{http://zruanweb.com/adh.zip}{here}. A function called
\verb|mktest()| will be used for the test.

%doctest examples lib:numpy
\begin{verbatim}
>>> from Bio import SeqIO, AlignIO
>>> from Bio.Alphabet import IUPAC
>>> from Bio.CodonAlign import build
>>> from Bio.CodonAlign.CodonAlignment import mktest

>>> pro_aln = AlignIO.read('adh.aln', 'clustal', alphabet=IUPAC.protein)
>>> p = SeqIO.index('drosophilla.fasta', 'fasta', alphabet=IUPAC.IUPACUnambiguousDNA())
>>> codon_aln = build(pro_aln, p)
>>> print(codon_aln)
CodonAlphabet(Standard) CodonAlignment with 27 rows and 768 columns (256 codons)
ATGGCGTTTACCTTGACCAACAAGAACGTGGTTTTCGTGGCCGGTCTGGGAGGCATT...ATC gi|9217|emb|X57365.1|
ATGGCGTTTACCTTGACCAACAAGAACGTGGTTTTCGTGGCCGGTCTGGGAGGCATT...ATC gi|9219|emb|X57366.1|
ATGGCGTTTACCTTGACCAACAAGAACGTGGTTTTCGTGGCCGGTCTGGGAGGCATT...ATC gi|9221|emb|X57367.1|
ATGGCGTTTACCTTGACCAACAAGAACGTGGTTTTCGTGGCCGGTCTGGGAGGCATT...ATC gi|9223|emb|X57368.1|
ATGGCGTTTACCTTGACCAACAAGAACGTGGTTTTCGTGGCCGGTCTGGGAGGCATT...ATC gi|9225|emb|X57369.1|
ATGGCGTTTACCTTGACCAACAAGAACGTGGTTTTCGTGGCCGGTCTGGGAGGCATT...ATC gi|9227|emb|X57370.1|
ATGGCGTTTACCTTGACCAACAAGAACGTGGTTTTCGTGGCCGGTCTGGGAGGCATT...ATC gi|9229|emb|X57371.1|
ATGGCGTTTACCTTGACCAACAAGAACGTGGTTTTCGTGGCCGGTCTGGGAGGCATT...ATC gi|9231|emb|X57372.1|
ATGGCGTTTACCTTGACCAACAAGAACGTGGTTTTCGTGGCCGGTCTGGGAGGCATT...ATC gi|9233|emb|X57373.1|
ATGGCGTTTACCTTGACCAACAAGAACGTGGTTTTCGTGGCCGGTCTGGGAGGCATT...ATC gi|9235|emb|X57374.1|
ATGGCGTTTACCTTGACCAACAAGAACGTGGTTTTCGTGGCCGGTCTGGGAGGCATT...ATC gi|9237|emb|X57375.1|
ATGGCGTTTACCTTGACCAACAAGAACGTGGTTTTCGTGGCCGGTCTGGGAGGCATT...ATC gi|9239|emb|X57376.1|
ATGGCGTTTACTTTGACCAACAAGAACGTGATTTTCGTTGCCGGTCTGGGAGGCATT...ATC gi|9097|emb|X57361.1|
ATGGCGTTTACTTTGACCAACAAGAACGTGATTTTCGTTGCCGGTCTGGGAGGCATT...ATC gi|9099|emb|X57362.1|
ATGGCGTTTACTTTGACCAACAAGAACGTGATTTTCGTTGCCGGTCTGGGAGGCATT...ATC gi|9101|emb|X57363.1|
ATGGCGTTTACTTTGACCAACAAGAACGTGATTTTCGTTGCCGGTCTGGGAGGCATC...ATC gi|9103|emb|X57364.1|
ATGTCGTTTACTTTGACCAACAAGAACGTGATTTTCGTGGCCGGTCTGGGAGGCATT...ATC gi|156879|gb|M17837.1|DROADHCK
ATGTCGTTTACTTTGACCAACAAGAACGTGATTTTCGTGGCCGGTCTGGGAGGCATT...ATC gi|156877|gb|M17836.1|DROADHCJ
ATGTCGTTTACTTTGACCAACAAGAACGTGATTTTCGTGGCCGGTCTGGGAGGCATT...ATC gi|156875|gb|M17835.1|DROADHCI
ATGTCGTTTACTTTGACCAACAAGAACGTGATTTTCGTGGCCGGTCTGGGAGGCATT...ATC gi|156873|gb|M17834.1|DROADHCH
ATGTCGTTTACTTTGACCAACAAGAACGTGATTTTCGTGGCCGGTCTGGGAGGCATT...ATC gi|156871|gb|M17833.1|DROADHCG
ATGTCGTTTACTTTGACCAACAAGAACGTGATTTTCGTTGCCGGTCTGGGAGGCATT...ATC gi|156863|gb|M19547.1|DROADHCC
ATGTCGTTTACTTTGACCAACAAGAACGTGATTTTCGTGGCCGGTCTGGGAGGCATT...ATC gi|156869|gb|M17832.1|DROADHCF
ATGTCGTTTACTTTGACCAACAAGAACGTGATTTTCGTGGCCGGTCTGGGAGGCATT...ATC gi|156867|gb|M17831.1|DROADHCE
ATGTCGTTTACTTTGACCAACAAGAACGTGATTTTCGTTGCCGGTCTGGGAGGCATT...ATC gi|156865|gb|M17830.1|DROADHCD
ATGTCGTTTACTTTGACCAACAAGAACGTGATTTTCGTTGCCGGTCTGGGAGGCATT...ATC gi|156861|gb|M17828.1|DROADHCB
ATGTCGTTTACTTTGACCAACAAGAACGTGATTTTCGTTGCCGGTCTGGGAGGCATT...ATC gi|156859|gb|M17827.1|DROADHCA

>>> pvalue = mktest([codon_aln[1:12], codon_aln[12:16], codon_aln[16:]])
>>> print(round(pvalue, 5))
0.00206
\end{verbatim}

In the above example, \texttt{codon\_aln{[}1:12{]}} belongs to
D. melanogaster, \texttt{codon\_aln{[}12:16{]}} belongs to D. simulans
and \texttt{codon\_aln{[}16:{]}} belongs to D. yakuba. \verb|mktest()|
will return the p-value of the test. We can see in this case, 0.00206
\textless{}\textless{} 0.01, therefore, the gene is under strong
negative selection according to MK test.

\section{Future Development}

Because of the limited time frame for Google Summer of Code project,
some of the functions in \verb|CodonAlign| is not tested
comprehensively. In the following days, I will continue perfect the code
and several new features will be added. I am always welcome to hear your
suggestions and feature request. You are also highly encouraged to
contribute to the existing code. Please do not hesitable to email me
(zruan1991 at gmail dot com) when you have novel ideas that can make the
code better.


% chapter19 Cookbook -- Cool things to do with it
\include{chapter/chapter19}

% chapter20 The Biopython testing framework
\include{chapter/chapter20}

% chapter21 Advanced
\include{chapter/chapter21}

% chapter22 Where to go from here -- contributing to Biopython
\include{chapter/chapter22}

% chapter23 Appendix: Useful stuff about Python
\chapter{Appendix: Useful stuff about Python}
\label{sec:appendix}

If you haven't spent a lot of time programming in Python, many
questions and problems that come up in using Biopython are often
related to Python itself. This section tries to present some ideas and
code that come up often (at least for us!) while using the Biopython
libraries. If you have any suggestions for useful pointers that could
go here, please contribute!

\section{What the heck is a handle?}
\label{sec:appendix-handles}

Handles are mentioned quite frequently throughout this documentation,
and are also fairly confusing (at least to me!). Basically, you can
think of a handle as being a ``wrapper'' around text information.

Handles provide (at least) two benefits over plain text information:

\begin{enumerate}
  \item They provide a standard way to deal with information stored in
  different ways. The text information can be in a file, or in a
  string stored in memory, or the output from a command line program,
  or at some remote website, but the handle provides a common way of
  dealing with information in all of these formats.

  \item They allow text information to be read incrementally, instead
  of all at once. This is really important when you are dealing with
  huge text files which would use up all of your memory if you had to
  load them all.
\end{enumerate}

Handles can deal with text information that is being read (e.~g.~reading
from a file) or written (e.~g.~writing information to a file). In the
case of a ``read'' handle, commonly used functions are \verb|read()|,
which reads the entire text information from the handle, and
\verb|readline()|, which reads information one line at a time. For
``write'' handles, the function \verb|write()| is regularly used.

The most common usage for handles is reading information from a file,
which is done using the built-in Python function \verb|open|. Here, we open a
handle to the file \href{examples/m\_cold.fasta}{m\_cold.fasta}
(also available online
\href{http://biopython.org/DIST/docs/tutorial/examples/m\_cold.fasta}{here}):

\begin{verbatim}
>>> handle = open("m_cold.fasta", "r")
>>> handle.readline()
">gi|8332116|gb|BE037100.1|BE037100 MP14H09 MP Mesembryanthemum ...\n"
\end{verbatim}

Handles are regularly used in Biopython for passing information to parsers.
For example, since Biopython 1.54 the main functions in \verb|Bio.SeqIO|
and \verb|Bio.AlignIO| have allowed you to use a filename instead of a
handle:

\begin{verbatim}
from Bio import SeqIO
for record in SeqIO.parse("m_cold.fasta", "fasta"):
    print(record.id, len(record))
\end{verbatim}

On older versions of Biopython you had to use a handle, e.g.

\begin{verbatim}
from Bio import SeqIO
handle = open("m_cold.fasta", "r")
for record in SeqIO.parse(handle, "fasta"):
    print(record.id, len(record))
handle.close()
\end{verbatim}

This pattern is still useful - for example suppose you have a gzip
compressed FASTA file you want to parse:

\begin{verbatim}
import gzip
from Bio import SeqIO
handle = gzip.open("m_cold.fasta.gz")
for record in SeqIO.parse(handle, "fasta"):
    print(record.id, len(record))
handle.close()
\end{verbatim}

See Section~\ref{sec:SeqIO_compressed} for more examples like this,
including reading bzip2 compressed files.

\subsection{Creating a handle from a string}

One useful thing is to be able to turn information contained in a
string into a handle. The following example shows how to do this using
\verb|cStringIO| from the Python standard library:

%doctest
\begin{verbatim}
>>> my_info = 'A string\n with multiple lines.'
>>> print(my_info)
A string
 with multiple lines.
>>> from StringIO import StringIO
>>> my_info_handle = StringIO(my_info)
>>> first_line = my_info_handle.readline()
>>> print(first_line)
A string
<BLANKLINE>
>>> second_line = my_info_handle.readline()
>>> print(second_line)
 with multiple lines.
\end{verbatim}

\begin{thebibliography}{99}
\bibitem{cock2009}
Peter J. A. Cock, Tiago Antao, Jeffrey T. Chang, Brad A. Chapman, Cymon J. Cox, Andrew Dalke, Iddo Friedberg, Thomas Hamelryck, Frank Kauff, Bartek Wilczynski, Michiel J. L. de Hoon: ``Biopython: freely available Python tools for computational molecular biology and bioinformatics''. {\it Bioinformatics} {\bf 25} (11), 1422--1423 (2009). \href{http://dx.doi.org/10.1093/bioinformatics/btp163}{doi:10.1093/bioinformatics/btp163},
\bibitem{pritchard2006}
Leighton Pritchard, Jennifer A. White, Paul R.J. Birch, Ian K. Toth: ``GenomeDiagram: a python package for the visualization of large-scale genomic data''.  {\it Bioinformatics} {\bf 22} (5): 616--617 (2006).
\href{http://dx.doi.org/10.1093/bioinformatics/btk021}{doi:10.1093/bioinformatics/btk021},
\bibitem{toth2006}
Ian K. Toth, Leighton Pritchard, Paul R. J. Birch: ``Comparative genomics reveals what makes an enterobacterial plant pathogen''. {\it Annual Review of Phytopathology} {\bf 44}: 305--336 (2006).
\href{http://dx.doi.org/10.1146/annurev.phyto.44.070505.143444}{doi:10.1146/annurev.phyto.44.070505.143444},
\bibitem{vanderauwera2009}
G\'eraldine A. van der Auwera, Jaroslaw E. Kr\'ol, Haruo Suzuki, Brian Foster, Rob van Houdt, Celeste J. Brown, Max Mergeay, Eva M. Top: ``Plasmids captured in C. metallidurans CH34: defining the PromA family of broad-host-range plasmids''.
\textit{Antonie van Leeuwenhoek} {\bf 96} (2): 193--204 (2009).
\href{http://dx.doi.org/10.1007/s10482-009-9316-9}{doi:10.1007/s10482-009-9316-9}
\bibitem{proux2002}
Caroline Proux, Douwe van Sinderen, Juan Suarez, Pilar Garcia, Victor Ladero, Gerald F. Fitzgerald, Frank Desiere, Harald Br\"ussow:
``The dilemma of phage taxonomy illustrated by comparative genomics of Sfi21-Like Siphoviridae in lactic acid bacteria''.  \textit{Journal of Bacteriology} {\bf 184} (21): 6026--6036 (2002).
\href{http://dx.doi.org/10.1128/JB.184.21.6026-6036.2002}{http://dx.doi.org/10.1128/JB.184.21.6026-6036.2002}
\bibitem{jupe2012}
Florian Jupe, Leighton Pritchard, Graham J. Etherington, Katrin MacKenzie, Peter JA Cock, Frank Wright, Sanjeev Kumar Sharma1, Dan Bolser, Glenn J Bryan, Jonathan DG Jones, Ingo Hein: ``Identification and localisation of the NB-LRR gene family within the potato genome''. \textit{BMC Genomics} {\bf 13}: 75 (2012).
\href{http://dx.doi.org/10.1186/1471-2164-13-75}{http://dx.doi.org/10.1186/1471-2164-13-75}
\bibitem{cock2010}
Peter J. A. Cock, Christopher J. Fields, Naohisa Goto, Michael L. Heuer, Peter M. Rice: ``The Sanger FASTQ file format for sequences with quality scores, and the Solexa/Illumina FASTQ variants''.  \textit{Nucleic Acids Research} {\bf 38} (6): 1767--1771 (2010). \href{http://dx.doi.org/10.1093/nar/gkp1137}{doi:10.1093/nar/gkp1137}
\bibitem{brown1999}
Patrick O. Brown, David Botstein: ``Exploring the new world of the genome with DNA microarrays''. \textit{Nature Genetics} {\bf 21} (Supplement 1), 33--37 (1999). \href{http://dx.doi.org/10.1038/4462}{doi:10.1038/4462}
\bibitem{talevich2012}
Eric Talevich, Brandon M. Invergo, Peter J.A. Cock, Brad A. Chapman: ``Bio.Phylo: A unified toolkit for processing, analyzing and visualizing phylogenetic trees in Biopython''.  \textit{BMC Bioinformatics} {\bf 13}: 209 (2012).  \href{http://dx.doi.org/10.1186/1471-2105-13-209}{doi:10.1186/1471-2105-13-209}
\bibitem{cornish1985}
Athel Cornish-Bowden: ``Nomenclature for incompletely specified bases in nucleic acid sequences: Recommendations 1984.'' \textit{Nucleic Acids Research} {\bf 13} (9): 3021--3030 (1985). \href{http://dx.doi.org/10.1093/nar/13.9.3021}{doi:10.1093/nar/13.9.3021}
\bibitem{cavener1987}
Douglas R. Cavener: ``Comparison of the consensus sequence flanking translational start sites in Drosophila and vertebrates.'' \textit{Nucleic Acids Research} {\bf 15} (4): 1353--1361 (1987). \href{http://dx.doi.org/10.1093/nar/15.4.1353}{doi:10.1093/nar/15.4.1353}
\bibitem{bailey1994}
Timothy L. Bailey and Charles Elkan: ``Fitting a mixture model by expectation maximization to discover motifs in biopolymers'', \textit{Proceedings of the Second International Conference on Intelligent Systems for Molecular Biology} 28--36. AAAI Press, Menlo Park, California (1994).
\bibitem{chapman2000}
Brad Chapman and Jeff Chang: ``Biopython: Python tools for computational biology''. \textit{ACM SIGBIO Newsletter} {\bf 20} (2): 15--19 (August 2000).
\bibitem{dehoon2004}
Michiel J. L. de Hoon, Seiya Imoto, John Nolan, Satoru Miyano: ``Open source clustering software''. \textit{Bioinformatics} {\bf 20} (9): 1453--1454 (2004). \href{http://dx.doi.org/10.1093/bioinformatics/bth078}{doi:10.1093/bioinformatics/bth078}
\bibitem{eisen1998}
Michiel B. Eisen, Paul T. Spellman, Patrick O. Brown, David Botstein: ``Cluster analysis and display of genome-wide expression patterns''. \textit{Proceedings of the National Academy of Science USA} {\bf 95} (25): 14863--14868 (1998). \href{http://dx.doi.org/10.1073/pnas.96.19.10943-c}{doi:10.1073/pnas.96.19.10943-c}
\bibitem{golub1971}
Gene H. Golub, Christian Reinsch: ``Singular value decomposition and least squares solutions''. In \textit{Handbook for Automatic Computation}, {\bf 2}, (Linear Algebra) (J. H. Wilkinson and C. Reinsch, eds), 134--151. New York: Springer-Verlag (1971).
\bibitem{golub1989}
Gene H. Golub, Charles F. Van Loan: \textit{Matrix computations}, 2nd edition (1989).
\bibitem{hamelryck2003a}
Thomas Hamelryck and  Bernard Manderick: 11PDB parser and structure class
implemented in Python''. \textit{Bioinformatics}, \textbf{19} (17): 2308--2310 (2003) \href{http://dx.doi.org/10.1093/bioinformatics/btg299}{doi: 10.1093/bioinformatics/btg299}. 
\bibitem{hamelryck2003b}
Thomas Hamelryck: ``Efficient identification of side-chain patterns using a multidimensional index tree''. \textit{Proteins} {\bf 51} (1): 96--108 (2003). \href{http://dx.doi.org/10.1002/prot.10338}{doi:10.1002/prot.10338}
\bibitem{hamelryck2005}
Thomas Hamelryck: ``An amino acid has two sides; A new 2D measure provides a different view of solvent exposure''. \textit{Proteins} {\bf 59} (1): 29--48 (2005). \href{http://dx.doi.org/10.1002/prot.20379}{doi:10.1002/prot.20379}.
\bibitem{hartigan1975}
John A. Hartiga. \textit{Clustering algorithms}. New York: Wiley (1975).
\bibitem{jain1988}
Anil L. Jain, Richard C. Dubes: \textit{Algorithms for clustering data}. Englewood Cliffs, N.J.: Prentice Hall (1988).
\bibitem{kachitvichyanukul1988}
Voratas Kachitvichyanukul, Bruce W. Schmeiser: Binomial Random Variate Generation. \textit{Communications of the ACM} {\bf 31} (2): 216--222 (1988). \href{http://dx.doi.org/10.1145/42372.42381}{doi:10.1145/42372.42381}
\bibitem{kohonen1997}
Teuvo Kohonen: ``Self-organizing maps'', 2nd Edition. Berlin; New York: Springer-Verlag (1997).
\bibitem{lecuyer1988}
Pierre L'Ecuyer: ``Efficient and Portable Combined Random Number Generators.''
\textit{Communications of the ACM} {\bf 31} (6): 742--749,774 (1988). \href{http://dx.doi.org/10.1145/62959.62969}{doi:10.1145/62959.62969}
\bibitem{majumdar2005}
Indraneel Majumdar, S. Sri Krishna, Nick V. Grishin: ``PALSSE: A program to delineate linear secondary structural elements from protein structures.'' \textit{BMC Bioinformatics}, {\bf 6}: 202 (2005). \href{http://dx.doi.org/10.1186/1471-2105-6-202}{doi:10.1186/1471-2105-6-202}.
\bibitem{matys2003}
V. Matys, E. Fricke, R. Geffers, E. G\"ossling, M. Haubrock, R. Hehl, K. Hornischer, D. Karas, A.E. Kel, O.V. Kel-Margoulis, D.U. Kloos, S. Land, B. Lewicki-Potapov, H. Michael, R. M\"unch, I. Reuter, S. Rotert, H. Saxel, M. Scheer, S. Thiele, E. Wingender E: ``TRANSFAC: transcriptional regulation, from patterns to profiles.'' Nucleic Acids Research {\bf 31} (1): 374--378 (2003). \href{http://dx.doi.org/10.1093/nar/gkg108}{doi:10.1093/nar/gkg108}
\bibitem{sibson1973}
Robin Sibson: ``SLINK: An optimally efficient algorithm for the single-link cluster method''. \textit{The Computer Journal} {\bf 16} (1): 30--34 (1973). \href{http://dx.doi.org/10.1093/comjnl/16.1.30}{doi:10.1093/comjnl/16.1.30}
\bibitem{snedecor1989}
George W. Snedecor, William G. Cochran: \textit{Statistical methods}. Ames, Iowa: Iowa State University Press (1989).
\bibitem{tamayo1999}
Pablo Tamayo, Donna Slonim, Jill Mesirov, Qing Zhu, Sutisak Kitareewan, Ethan Dmitrovsky, Eric S. Lander, Todd R. Golub: ``Interpreting patterns of gene expression with self-organizing maps: Methods and application to hematopoietic differentiation''. \textit{Proceedings of the National Academy of Science USA} {\bf 96} (6): 2907--2912 (1999). \href{http://dx.doi.org/10.1073/pnas.96.6.2907}{doi:10.1073/pnas.96.6.2907}
\bibitem{tryon1970}
Robert C. Tryon, Daniel E. Bailey: \textit{Cluster analysis}. New York: McGraw-Hill (1970).
\bibitem{tukey1977}
John W. Tukey: ``Exploratory data analysis''. Reading, Mass.: Addison-Wesley Pub. Co. (1977).
\bibitem{yeung2001}
Ka Yee Yeung, Walter L. Ruzzo: ``Principal Component Analysis for clustering gene expression data''. \textit{Bioinformatics} {\bf 17} (9): 763--774 (2001). \href{http://dx.doi.org/10.1093/bioinformatics/17.9.763}{doi:10.1093/bioinformatics/17.9.763}
\bibitem{saldanha2004}
Alok Saldanha: ``Java Treeview---extensible visualization of microarray data''. \textit{Bioinformatics} {\bf 20} (17): 3246--3248 (2004). 
\href{http://dx.doi.org/10.1093/bioinformatics/bth349}{http://dx.doi.org/10.1093/bioinformatics/bth349}
\end{thebibliography}


\end{document}
